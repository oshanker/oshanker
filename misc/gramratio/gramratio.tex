%%%%%%%%%%%%%%%Conjectures regarding values of Riemann zeta function at Gram points%%%%%%%%%%%%%%%%%%%%%%%%%%%%%%%%%%%%%%%%%%%%%%%%%%%%%%%%%%%%%
%

\documentclass[twoside]{article}
\usepackage{graphicx}
\usepackage{amsmath,amsthm,amssymb,verbatim}
\usepackage{fancyhdr}
\pagestyle{fancy}
\usepackage{url}

\def\blfootnote{\xdef\@thefnmark{}\@footnotetext} 
\long\def\symbolfootnote[#1]#2{\begingroup%
\def\thefootnote{\fnsymbol{footnote}}\footnote[#1]{#2}\endgroup} 

\newtheorem{mydef}{Conjecture}
\newtheorem*{mydef-non}{Conjecture}

\theoremstyle{definition}
\newtheorem{defn}{Definition}

\setcounter{page}{1}
\begin{document}

\date{}
\lhead[]{}
\chead[]{}
\rhead[]{}

\title{\bf{Good to Bad Gram point Ratio For Riemann Zeta Function}}
%

\author{O. Shanker 
 \thanks{Mountain View, CA 94041, U. S. A. Email: oshanker@gmail.com
 }
}

\maketitle
\thispagestyle{fancy}

\begin{abstract}
In this work we consider the asymptotic value of the ratio of good to bad Gram points for the Riemann zeta function.
We present two new results. The first new result is a relation between the ratio of good to bad Gram points and the distribution of Gram intervals that contain a given number of zeros.
We relate this to a conjecture of Odlyzko about the locations of the zeros of the Riemann zeta function.
If the conjecture is correct, we show that the ratio of good to bad Gram points should be $1$. 
We show that the empirical evidence does not favor Odlyzko's conjecture. The  second new result is the formulation and empirical validation of two 
symmetry related conjectures about the location of the zeros.
\end{abstract}



%\clearpage
\chead[\underline{O. Shanker}]{\underline{Good to Bad Gram point Ratio For Riemann Zeta Function}}


\section{Introduction}
The Riemann Hypothesis is one of the seven millennium prize problems posed by the Clay Mathematical Institute in 2000~\cite{Sarnak 2005}. 
The earliest detailed evaluations of the roots of the Riemann zeta function were carried out in 1903 by Gram~\cite{Gram 1903}.
In evaluating the zeros he made use of the values of the Riemann zeta function at special (very regularly placed) points which are called ``Gram points".
The Gram points lie on the line $1/2+it$ in the complex plane, and the Riemann zeta function $\zeta(1/2+it)$ is real at these points. 
Gram observed that in his calculations the roots of the Riemann zeta function alternated with the Gram points. He also observed that the
value of the real part of the Riemann zeta function tends to be positive at the Gram points. Such points are today called good Gram points. 
\begin{defn}\label{good0}
A Gram point is called good if the real part of the Riemann zeta function is positive at the Gram point.
\end{defn}
Thus, we may say that since the early 
twentieth century we have been observing that the majority of Gram points are good Gram points. 

In this work we study empirically the ratio of good to bad Gram points, and relate it to Odlyzko's~\cite{Odlyzko 1992} conjecture,  that at large heights a Gram interval does not differ from any other interval of that length. Odlyzko used this conjecture, and the GUE hypothesis, to derive the distribution of Gram intervals that contain a given number of zeros.
If the conjecture is correct, we show that the ratio of good to bad Gram points should be $1$. This is in conflict with the empirical observation that good Gram points tend to be more numerous than bad Gram points. In Section~\ref{sec5} we derive a new relation between the ratio of good to bad Gram points and the distribution of Gram intervals that contain a given number of zeros.  Based on the empirical data we also formulate two new symmetry related conjectures about the location of the zeros, as stated below and explained in Section~\ref{sec7}.

\begin{mydef-non}\label{antisymmetry0}
(even-odd antisymmetry): The distribution of the zeta values for odd Gram points is the negative of the distribution of zeta values for the even Gram points.
\end{mydef-non}
\begin{mydef-non}\label{symmetry0}
(forward-backward symmetry): When we consider a sequence of zeta values at consecutive Gram points, the properties of the sequence are symmetric with respect to the direction of the sequence of Gram points (i.e., the sequence behaves similarly whether we consider the points in increasing order or in decreasing order).
\end{mydef-non}

The paper is organized as follows.
Section~\ref{sec2} establishes the required notation for the 
Riemann Zeta Function. 
Section~\ref{sec3} describes the Gram points. 
Section~\ref{sec4} describes the Gram blocks. 
In Section~\ref{sec5} we study empirically the ratio of good to bad Gram points, and relate it to a conjecture of Odlyzko. 
In Section~\ref{sec6} we study the ratio of $Type~II/Type~I$ Gram blocks. This ratio, and other ratios of more complex Gram block patterns, show a
remarkable structure, i.e., the ratios are very close to $1$ for Gram blocks of different lengths.
Section~\ref{sec7} presents two new conjectures based on the empirical data. 
Section~\ref{conclusions} gives a brief summary of the results. 

\section{\label{sec2}The Notation for the Riemann zeta function }

In this section we  establish the required notation for the 
Riemann Zeta Function. 
The Riemann Zeta function is defined for $\mathrm{Re} (s) > 1$ by
\begin{equation}
\zeta ( s ) \, = \, \sum^{\infty}_{n = 1} \; n^{-s} \, = \, \prod_{p \in primes} \;
\left( 1 - p^{-s} \right)^{-1}.
\label{eqRie}
\end{equation}

Eq.~(\ref{eqRie})  converges for $\mathrm{Re} (s) > 1$.  
 $\zeta ( s )$ has a  continuation
to the complex plane and satisfies a functional equation \cite{Riemann(1858),Riemann 1892, Titchmarsh 1986,Edwards(1974)}
\begin{equation}  
\xi(s):=s(s-1) \pi^{-s/2} \, \Gamma (s/2) \, \zeta ( s )/2 \, = \, \xi ( 1 - s );
\label{eq:xifunc}
\end{equation}
$\xi(s)$ is an entire function. We
write the zeroes of $\xi(s)$ as $1/2 + i \gamma$. The Riemann Hypothesis  
asserts that $\gamma$ is real for the non-trivial zeroes.
We order the $\gamma$s in increasing order, with 
\begin{equation}
\ldots \ldots \gamma_{-1} \, < \, 0 \, < \, 
\gamma_1 \, \leq \, \gamma_2 \ldots. 
\end{equation}
Then $\gamma_j \, = \, - \gamma_{-j}$ for $j = 1, 2, \ldots,$ 
and    $\gamma_1$, $\gamma_2$, $\ldots$  are roughly
$14.1347$, $21.0220$, $\ldots$.


Asymptotically, for the Riemann zeta function the mean number of 
zeros with height less than $\gamma$ (the smoothed Riemann zeta staircase, which we denote by $N(\gamma) $)
is~\cite{Edwards(1974)}
\begin{equation}  
N(\gamma) = (\gamma/2\pi)(ln(\gamma/2\pi)-1)+\frac{7}{8}.
\label{eq:Rnumber}
\end{equation}
Thus, the mean spacing of the zeros at height $\gamma$ is 
$2\pi(\ln (\gamma/2\pi))^{-1}$. 

The study of the zeroes of the Riemann zeta function and Generalized 
Zeta functions is of interest to mathematicians and physicists. Mathematicians 
study the spacings because of its applications to analytic number theory, 
while physicists study it because of its  relation 
to the theory of the spectra of random matrix theories (RMT) 
and the spectra of classically chaotic quantum systems. 
Many remarkable properties of the Riemann zeta function keep turning up in the literature~\cite{os6,Matiyasevich}.

\section{\label{sec3}Gram points}

In this section we discuss the details of the numerical work. 
The numerical analysis takes advantage of the functional 
equation Eq.~(\ref{eq:func}).
One defines
\begin{equation}
\theta(t) = arg (\pi^{it/2} \Gamma(\frac{1}{4} + \frac{it}{2})), 
\label{eq:theta}
\end{equation}
where the argument is defined by continuous variation of $t$ starting with the value $0$ at $t = 0$.
For large $t$, $\theta$ has the asymptotic expansion
\begin{equation}
\theta(t) \approx \frac{t}{2}\ln (\frac{t}{2\pi}) - \frac{t}{2} - \frac{\pi}{8} + \frac{1}{48t} - \frac{1}{5760t^3}. 
\label{eq:thetaAsymptotic}
\end{equation}

A consequence of the zeta functional equation is that the function 
$Z(t)=exp(i\theta(t))\zeta(1/2 +it)$ is real valued for real $t$. 
Moreover we have $|Z(t)| = |\zeta(1/2+it)|$. Thus the zeros of $Z(t)$ are the imaginary part of the zeros 
of $\zeta(s)$ which lie on the critical line. We are led to finding the change of sign of a real valued function 
to find zeros on the critical line. This is a very convenient property in the numerical verification 
of the Riemann Hypothesis. Another very helpful property is that many of the zeros are separated by the
``Gram points".  When $t \ge 7$, the $\theta$ function Eq.(\ref{eq:theta}) is monotonic increasing. 
For $n \ge -1$, the $n$-th Gram point $g_n$ is defined as the unique solution $> 7$ to
$\theta (g_n) = n\pi$. Thus, at a Gram point we have
\begin{equation}
\zeta(1/2+ig_n) = (-1)^{n}Z(g_n).
\label{eq:zetagram}
\end{equation}

The Gram points are as dense as the zeros of $\zeta(s)$ but are much more regularly distributed.
Their locations can be found without any evaluations of the Riemann-Siegel series Eq.(\ref{eq:RS}).
Gram's law is the empirical observation that $Z(t)$ usually changes its sign in each Gram interval 
$G_n = [g_n,g_{n+1})$. 
This law fails infinitely often, but it is true in a large proportion of cases.
The average value of $Z(g_n)$ is $2$ for even $n$ and $-2$ for odd $n$~\cite{Titchmarsh 1986},
and hence $Z(g_n)$ undergoes an infinite number of sign changes.

The $Z$ function is evaluated using the Riemann$-$Siegel series
\begin{equation}
Z(t) = 2\sum^{m}_{n=1}\frac{\cos(\theta(t) - t \ln (n))}{\sqrt{n}} + R(t), 
\label{eq:RS}
\end{equation}
where $m$ is the integer part of $\sqrt{t/(2\pi)}$, and $R(t)$ is a small remainder
term which can be evaluated to the desired level of accuracy. The most important 
source for loss of accuracy at large heights is the cancellation between
large numbers that occur in the arguments of the $\cos$ terms in Eq.~(\ref{eq:RS}). We 
use a high precision module to evaluate the arguments. The rest of the calculation
is done using regular double precision accuracy. See ~\cite{hiary,gourdon,Odlyzko(1989)} for methods to efficiently evaluate the zeta function at large $t$.

\section{\label{sec4}Gram blocks}


In this section we present the different types of Gram blocks. We will follow the notation of   Odlyzko~\cite{Odlyzko 1992}. A common definition of a good Gram point is
\begin{defn}\label{good1}
A Gram point $g_n$ is called good if $(-1)^nZ(g_n) > 0$, and bad otherwise.
\end{defn}

Definitions \ref{good0} and \ref{good1} are equivalent because of Eq.~\ref{eq:zetagram}.
A Gram block is an interval $[g_n, g_{n+k})$ such that $g_n$  and $g_{n+k}$ are good Gram points 
and $g_{n+1}, . . ., g_{n+k-1}$ are bad Gram points. A Gram block is denoted by the notation $a_1a_2 . . . a_k$ where $k$ is called the length of the Gram block, and $a_i$ denote the number of roots of $Z(t)$ in the Gram interval $[g_{n+i-1}, g_{n+i})$. So far, no Gram interval has been found with more than 5 zeros, thus the notation is unambiguous. $a_1$ and $a_k$ must be even while  $a_2$ to $a_{k-1}$ are odd.

A Gram block of length $k$ which contains exactly $k$ roots of $Z[t)$ is called regular. The first and last Gram intervals of a regular Gram block must contain an even number of roots (0 or 2 roots). 
The internal Gram intervals must all contain an odd number of roots (all of them must contain one root if the end intervals contain 2 and 0 roots. If the end intervals both contain no roots, then one of the internal intervals must contain 3 roots.) 
Thus, regular Gram blocks must have a pattern of one of the following three forms:
\begin{eqnarray}
21 . . . 10,\nonumber\\
 01 . . . 12,\nonumber \\
 01 . . . 131 . . . 10
\label{types}
\end{eqnarray}
where the notation $1 . . . 1$ refers to any string of consecutive 1s, including the zero length string. Odlyzko~\cite{Odlyzko 1992} denotes these three types of regular Gram blocks as Type I, Type II and Type III respectively.  A regular Gram block of length 1 is a special case and is not covered by any of the three forms stated above. Note that the sequence of zero counts in a Type II Gram block is the reverse of the sequence of zero counts in a Type I Gram block of the same length. We use this observation, and the results of Section~\ref{sec6}, where we study the ratio of $Type~II/Type~I$ Gram blocks, as part of the evidence for the new conjectures (in particular,  Conjecture~\ref{symmetry}) that we present in Section~\ref{sec7}.

\section{\label{sec5}Good to Bad Gram point ratio}

\begin{table}
\centering \(\begin{array}{llllll}
\hline
Source &m = 0&m = 1&m = 2&m = 3&size\\
\hline
t = 10^{12}&0.148797&0.704395&0.144823&0.001987&1000000\\
t = 10^{15}&0.1536533&0.6946820&0.1496761&0.0019886&10000000\\
t = 10^{28}&0.1619184&0.6781499&0.1579456&0.0019855&10000000\\
Odlyzko&0.170222&0.661430&0.166490&0.001860&\\
\hline
\end{array}\)
\caption{Counts of Gram intervals that contain $m$ zeros, for three samples at $t=10^{12}$, $t=10^{15}$  and $t=10^{28}$ respectively, and the  expected values using Odlyzko's prediction~\cite{Odlyzko 1992}.} \label{tab:intervalzeros}
\end{table}

\begin{table}
\centering \(\begin{array}{ccccccc}
\hline
Source &p_g&p_b& p_g/p_b&p_{even|bad} &p_{even|good}&\frac{p_{even|bad}}{p_{even|good}}\\
\hline
t = 10^{12}&0.7962&0.2038&3.9&0.7204&0.1844&3.9\\
t = 10^{15}&0.7792&0.2208&3.5&0.6868&0.1946&3.5\\
t = 10^{28}&0.7374&0.2626&2.8&0.6090&0.2169&2.8\\
\hline
\end{array}\)
\caption{Empirical verification of  Eq.~\ref{eq:consistency} for three samples at $t=10^{12}$, $t=10^{15}$  and $t=10^{28}$. Also note that Gram's law (i.e., $p_g > p_b$)  continues to hold
at least upto  $t = 10^{28}$.} 
\label{tab:pevenpred}
\end{table}

In this section we consider the distribution of Gram intervals that contain a given number of zeros. We present a new relation between the distribution and the good/bad nature of Gram points. 
We  review the prediction of Odlyzko~\cite{Odlyzko 1992} regarding the distribution of Gram intervals that contain a given number of zeros. The prediction is based on the conjecture that at large heights a Gram interval does not differ from any other interval of that length. Odlyzko used this conjecture, and the GUE hypothesis, to derive the distribution of Gram intervals that contain a given number of zeros.  
The GUE hypothesis  is the hypothesis that the distribution of the normalized spacing between zeros of the Zeta function is asymptotically equal to the distribution of the eigenvalues of random hermitian matrices with independent normal distribution of its coefficients. Such random hermitian matrices form the Gauss unitary ensemble (GUE). Under the assumptions of Odlyzko the distribution of Gram intervals that contain a given number of zeros is given by the probability that an interval of length equal to the Gram interval contains exactly $m$ zeros. In what follows we will refer to this distribution as Odlyzko's prediction.

Table~\ref{tab:intervalzeros} shows the counts of Gram intervals that contain $m$ zeros, for three samples at $t=10^{12}$, $t=10^{15}$  and $t=10^{28}$ respectively. The last row of the table shows the  expected values for the counts from Odlyzko's prediction. For calculating the first three rows of the table we just need a tabulation of the zeros at the respective heights. The Gram points can be located using Eq.~\ref{eq:thetaAsymptotic}. We use the zeros from Ref~\cite{hiary 2010}. The last row is from Ref.~\cite{Odlyzko 1992}.

While the agreement with Odlyzko's prediction is good, we argue here that the distribution of zero counts in Gram intervals is not independent of the type of Gram interval, and that it has to depend on whether the left side Gram point of the interval is good or bad. We reach this conclusion by considering a self-consistency condition between the probability of a Gram point being good or bad, and the probability that the corresponding interval contains an even or odd number of zeros of $Z(t)$.  

We set up the notation. We consider an interval  at height $t$ (our sample space) which is large compared to the Gram interval, but small enough that we can consider $ln(t)$ to be essentially constant over the interval.  Thus, the interval contains a large number of Gram points. Let $p_g$ be the ratio of the number of good Gram points in the interval to the total number of Gram points in the interval. Similarly, let $p_b$ be the ratio of the number of bad Gram points in the interval to the total number of Gram points in the interval. $p_g$ and $p_b$ represent the probabilities that a given Gram point is good or bad respectively.  

In the same sample space as in the above paragraph, we define the conditional probabilities that a Gram interval contains an even or odd number of roots, given that the left Gram point of the Gram interval is good or bad respectively. Let us first consider all Gram intervals which have a good Gram point on the left side. Let $p_{odd|good}$ be the ratio of the number of  Gram intervals in the sample space that have a left hand good Gram point and an odd number of roots, to the total number of Gram intervals in the sample space which have a left hand good Gram point. Let $p_{even|good}$ be the ratio of the number of  Gram intervals in the sample space that have a left hand good Gram point and an even number of roots, to the total number of Gram intervals in the sample space which have a left hand good Gram point. $p_{odd|good}$ and $p_{even|good}$ are the conditional probabilities that a Gram interval contains an odd or even number of zeros respectively, given that the left Gram point of the Gram interval is good. From the definition it follows that $p_{odd|good} + p_{even|good} = 1$. $p_{odd|bad}$ and $p_{even|bad}$ have corresponding interpretations when the left Gram point is bad. We now derive a consistency relation between these quantities, by asking the question: what is the probability of a given Gram point being good or bad, given information about the preceding Gram point? 

\begin{eqnarray}
&&p_g*p_{odd|good}  + p_b*p_{even|bad}\, =  p_g,\nonumber\\
&&p_g*p_{even|good} + p_b*p_{odd|bad}\, = p_b.
\label{eqGoodRelation}
\end{eqnarray}

The relationship between the quantities is shown in Eq.~\ref{eqGoodRelation}. The equation follows from the following simple facts. A given Gram point will be good if the preceding Gram point is good, and the intervening interval contains an odd number of zeros, or if the preceding Gram point is bad, and the intervening interval contains an even number of zeros. A given Gram point will be bad if the preceding Gram point is good, and the intervening interval contains an even number of zeros, or if the preceding Gram point is bad, and the intervening interval contains an odd number of zeros. This is is shown in Eq.~\ref{eqGoodRelation}. This is an eigenvalue equation for  $p_g$ and $p_b$. The condition for this equation to have a non-trivial solution for  $p_g$ and $p_b$ is
\begin{equation}
p_b*p_{even|bad} = p_g*p_{even|good}.
\label{eq:consistency}
\end{equation}
Table~\ref{tab:pevenpred} shows the empirical data which validates this relationship. Eq.~\ref{eq:consistency} can also be stated as $p_g/p_b = p_{even|bad} /p_{even|good}$.

Odlyzko's conjecture,  that at large heights a Gram interval does not differ from any other interval of that length, implies that $p_{odd|good} = p_{odd|bad}$ and  $p_{even|good} = p_{even|bad}$. If this were indeed true, then Eq.~\ref{eqGoodRelation} can only be satisfied for $p_g = p_b = 0.5$ . Empirically, we know that $p_g > p_b$ (which is Gram's law). Thus, Gram's law provides empirical evidence against Odlyzko's conjecture. 
Is it possible that the conjecture is asymptotically true, i.e., asymptotically, $p_g = p_b = 0.5$? The data for  $p_g$ and $p_b$ in Table~\ref{tab:pevenpred}  show that at least for $t$ of this order,  the  evidence does not favor the conjecture. 

\section{\label{sec6}Ratio of $Type~II/Type~I$ Gram block counts}

\begin{table}
\centering \(\begin{array}{cc}
\hline
Length~of 	& Ratio  \\
Gram~block	& Type~II/Type~I \\
\hline
2& 0.999966866\\
3& 0.999941442\\
4& 0.999913803\\
5& 0.999871412\\
6& 0.999822002\\
7& 0.999594807\\
8& 0.999257121\\
9& 1.001677139\\
10& 0.992668154\\
\hline
\end{array}\)
\caption{Equality of $Type~II$ and $Type~I$ Gram block counts for Gram blocks of different lengths. The statistics are from the first $10^{13}$ Gram intervals. The counts include only regular Gram blocks, since $Type~II$ and $Type~I$ Gram blocks are defined only for regular Gram blocks.} \label{tab:rosser}
\end{table}

\begin{table}
\centering \(\begin{array}{cccccc}
\hline
Length~of 	& &&&Displacement \\
Gram~block	& -0.2\delta & -0.1\delta & 0.0\delta & 0.1\delta & 0.2\delta  \\
\hline
2 &2.268576&1.504897&0.999090&0.663992&0.441400 \\
3 &3.624520&1.896329&0.998864&0.526210&0.274796 \\
4 &5.588850&2.367189&1.001220&0.426549&0.178314 \\
5 &8.849518&2.923269&1.011907&0.343228&0.115100 \\
6 &14.373004&3.728921&1.008597&0.266224&0.070801 \\
7 &23.961623&4.721631&0.974761&0.205256&0.041497 \\
8 &51.790514&6.714706&0.983726&0.149338&0.018210 \\
9 &116.632353&10.129730&0.975052&0.103749&0.008631 \\
10 &361.615385&13.306122&0.949264&0.057506&0.004515 \\
11 &1397.000000&34.533333&1.027888&0.041000&0.001084 \\
\hline
\end{array}\)
\caption{Test that the equality of $Type~II$ and $Type~I$ Gram block counts are not just a result of randomness over and above well-known distribution. The table shows the ratio of $Type~II/Type~I$ counts when we displace the Gram points by $n\delta$, where $\delta$ is the Gram interval. The statistics are from $10$ million Gram intervals at $t=10^{28}$.} \label{tab:rosserrandom}
\end{table}

\begin{table}
\centering \(\begin{array}{cccccc}
\hline
Length~of 	& &&Displacement \\
Gram~block	& -0.2\delta & -0.1\delta & 0.0\delta & 0.1\delta & 0.2\delta  \\
\hline
4 &0.632816&0.829027&1.011644&1.248305&1.463930 \\
5 &0.409091&0.642699&0.977058&1.503534&2.505236 \\
6 &0.263617&0.474359&0.987097&1.625000&2.926174 \\
7 &0.207627&0.489247&1.029412&2.359551&5.361702 \\
8 &0.133803&0.385321&0.870130&1.895833&6.222222 \\
9 &0.032258&0.290909&1.172414&2.600000&25.666667 \\
10 &0.068966&0.424242&1.000000&5.833333&25.500000 \\
11 &0.068966&0.142857&1.000000&10.500000& \\
\hline
\end{array}\)
\caption{Test that the equality of  $Type~III,left$ and $Type~III,right$ Gram block counts are not just a result of randomness over and above well-known distribution. The table shows the ratio of $Type~III,left/Type~III,right$ counts when we displace the Gram points by $n\delta$, where $\delta$ is the Gram interval. The statistics are from $10$ million Gram intervals at $t=10^{28}$.} \label{tab:rosser3random}
\end{table}


\begin{figure}
\centering
\includegraphics[width=1.1\textwidth]{typeIIratio}
\caption[]{ 
  Ratio of $Type~II/Type~I$ counts for Gram blocks of various lengths $l$, when we displace the Gram points by $n\delta$, where $\delta$ is the Gram interval. All the curves cross over exactly at the Gram point, and have the value $1$ at crossover. Thus, the forward-backward symmetry depends strongly on measuring at the Gram point.
 }
\vspace{1mm}
\label{typeIIratio}
\end{figure}


In this section we investigate further the assumption that that a Gram interval does not differ from any other interval of that length. 

lead up to this. why is this important? what is striking about the relationships?

We study this in the context of the ratio of $Type~II/Type~I$ counts in a given large interval. The starting point is the observation of a remarkable relation, namely, the ratio is very close to one for Gram blocks of different lengths (the relation excludes the case of regular Gram blocks of length 1, which do not fit into one of the patterns of Eq.~\ref{types}).   Table~\ref{tab:rosser}  (based on data in \cite{gourdon}) shows this fact for all zeros up to $t = 10^{13}$. One question we can ask is the extent to which the results follow just from randomness over and above the basic known patterns of the Riemann zeta distribution. In other words, do the results follow just from the well-known (conjectured) distribution of zero differences on the critical line? Would the results change significantly if we shift all the Gram points  by the same amount in a large interval? We address this question in this section. Table~\ref{tab:rosserrandom} and Figure~\ref{typeIIratio} shows the ratio of $Type~II/Type~I$ counts when we displace the Gram points by $n\delta$, where $\delta$ is the Gram interval. The statistics are from $10$ million Gram intervals at $t=10^{28}$.
They  show clearly that the properties are indeed strongly correlated with the Gram points, and hence are not just manifestations of randomness over and above the basic known patterns of the Riemann zeta distribution. 
We can further study the forward-backward symmetry (and the check that the ratio does depend on whether we measure from Gram points, or from points displaced from Gram points), using the ratio of  $Type~III,left$ interval counts where the '3' interval occurs at the first odd interval position on the left, to  $Type~III,right$ interval counts where the '3' interval occurs at the last odd interval position on the right. 
Table~\ref{tab:rosser3random} shows the ratio of $Type~III,left/Type~III,right$ counts when we displace the Gram points by $n\delta$, where $\delta$ is the Gram interval. This provides further evidence for the conclusions of Table~\ref{tab:rosserrandom}.
  
\begin{table}
\centering \(\begin{array}{cc}
\hline
Type~of~violation &Ratio~of~counts\\
\hline
2L3/2R3 &0.999651352\\
2L22/2R22 &0.999806111\\
3L3/3R3 &0.999217106\\
2L212/2R212 &0.999160495\\
3L22/3R22 &0.999640429\\
4L3/4R3 &0.998017358\\
2L2112/2R2112 &0.998100056\\
2L032/2R230 &0.999734774\\
3L212/3R212 &0.995621266\\
4L22/4R22 &0.998245284\\
2L04/2R40 &0.998111916\\
\hline
\end{array}\)
\caption{Forward-backward symmetry in patterns of violations of Rosser's rule.  The statistics are from the first $10^{13}$ Gram intervals.} \label{tab:vrr}
\end{table}

Further validation comes from a consideration of the different types of violations of Rosser's rule~\cite{gourdon}, which also show the forward-backward symmetry. Rosser's rule states that Gram blocks of length $k$ contain at least $k$ zeros. This law is violated infinitely often, but is violated only for a small fraction of the Gram blocks. A Gram block is either regular, or it has an excess of zeros, or it has fewer zeros than its length. The last type of Gram block violates Rosser's rule. 
If a Gram block of length k is an exception to Rosser's rule, then its pattern of zeros must be of the form $01...10$. To describe the exception, we must specify where the two missing zeros are. Odlyzko uses the notation $kXa_1a_2 . . . a_m, X = L~or~R $ 
to describe an exception on a Gram block of length $k$  where the missing zeros are on the left (for $X = L$) or on the right (for $X = R$), the pattern containing the missing zeros being $a_1a_2 . . . a_m$ (moreover, this pattern is the smallest union of Gram block adjacent to the exception that contains the missing zeros). For example, $3L04$ denotes a violation of Rosser's rule on a Gram block of length 3, the missing zeros being at its left. Written out in detail, the pattern of zeros is expressed by the notation is $04010$.

The notation above classifies the type of violation of Rosser's rule, the value $m$ being called the length of the excess block. The notation used for exceptions to Rosser's rule is not unambiguous. When several contiguous violations of Rosser's rule exists, they may overlap or missing zeros can be in the same Gram interval. Such situations are very rare, and in these cases (Gourdon found just three occurrences until the $10^{13}th$ zero), Odlyzko uses the notation $Ma_1 ...a_l$ where the pattern $a_1 . . . a_l$ is made of the minimal contiguous Gram blocks containing at least one violation to Rosser rule, and all the missing zeros. For example, the pattern $M00500$, first encountered at gram index $n = 3,680,295,786,518$, denotes a situation with two violations of Rosser's rule ($00$ and $00$, Gram blocks with missing zeros) and a single Gram interval containing all the missing zeros (pattern $5$). 
Table~\ref{tab:vrr}  (based on data in \cite{gourdon}) shows the forward-backward symmetry in patterns of violations of Rosser's rule. Once again, while there is very good evidence for the forward-backward symmetry, the $L$ type violations are ever so slightly smaller than the $R$ type violations. Again, we have evidence for a symmetry which is broken very slightly.

\section{\label{sec7}Two Conjectures}

\begin{figure*}
\centering
\includegraphics[width=0.62\textwidth]{ozeta.jpg}
\caption[]{ 
  Distribution of zeta values at 500000 odd Gram points  at $t = 10^{12}$.
 }
\vspace{1mm}
\label{oddhist}

\includegraphics[width=0.62\textwidth]{ezeta.jpg}
\caption[]{ 
   Distribution of zeta values at 500000 even Gram points  at $t = 10^{12}$.
 }
\label{evenhist}
\vspace{1mm}
\end{figure*}

\begin{table}
\centering \(\begin{array}{ccccccc}
\hline
 Gram &     Min.   & 1st    &  Median    &   Mean   & 3rd    &   Max. \\
 type &              & Quantile   &            &              & Quantile.    &   \\
\hline
All& -160.90 &   -1.17 &    0.00106 &   0.00  &  1.172 &165.10\\
Odd&-160.90 &   -2.526 &   -0.8471  & -2.00 &   -0.1121 &  69.41 \\
Even&-68.63 &   0.1139 &  0.8526  & 2.00 &   2.541 & 165.10 \\
\hline
\end{array}\)
\caption{Quantiles and mean for zeta values at Gram points of different types.  The statistics are from $1$ million Gram intervals at $t=10^{12}$.} \label{tab:quantiles}
\end{table}


\begin{table}
\centering \(\begin{array}{ccccc}
\hline
 Gram~type&   --   & -+   & +-   & ++  \\
\hline
Odd & 0.151651&0.627465&0.069119&0.151765 \\
Even & 0.151565&0.069206&0.627551&0.151678 \\
\hline
\end{array}\)
\caption{Counts of different configurations of zeta values for pairs of consecutive Gram points.  The statistics are from $10$ million Gram intervals at $t=10^{15}$.} \label{tab:pairraw}
\end{table}

\begin{table}
\centering \(\begin{array}{ccccc}
\hline
 Gram~type&   --   & -+   & +-   & ++  \\
\hline
Odd & 0.151651&0.627465&0.069119&0.151765\\
\hline
Corresponding& 0.151678 & 0.627551 & 0.069206& 0.151565\\ 
Even~Entry     & (++)     & (+-)   & (-+)  & (--) \\
\hline
\end{array}\)
\caption{Test of Conjecture \ref{antisymmetry} using pairs of consecutive Gram points.  The statistics are from $10$ million Gram intervals at $t=10^{15}$.} \label{tab:pairtest}
\end{table}


The empirical data presented in the previous section lead us to formulate two new conjectures  about the distribution of zeta values at the 
the Gram points. These conjectures are most likely related to the symmetry properties of the rotated zeta function (Hardy's function) under inversion around the origin. 

Why is this striking?

\begin{mydef}\label{antisymmetry}
(even-odd antisymmetry): The distribution of the zeta values for odd Gram points is the negative of the distribution of zeta values for the even Gram points.
\end{mydef}
\begin{mydef}\label{symmetry}
(forward-backward symmetry): When we consider a sequence of zeta values at consecutive Gram points, the properties of the sequence are symmetric with respect to the direction of the sequence of Gram points (i.e., the sequence behaves similarly whether we consider the points in increasing order or in decreasing order).
\end{mydef}

The evidence for  Conjecture~\ref{symmetry} has been presented in the previous sections. 
In this section we will consider the statistics of zeta values at pairs of consecutive Gram points. We will look at the evidence for Conjecture \ref{antisymmetry}.  All the data  are from $10$ million Gram intervals starting at $t=10^{15}$.

Figures \ref{oddhist} and ~\ref{evenhist}  present the histograms of zeta values at odd Gram points and even Gram points respectively. These figures further bear out the evidence of Table~\ref{tab:quantiles} that the odd and even distributions are mirror images of each other. To study Conjecture \ref{antisymmetry} further, we will consider the zeta values at pairs of consecutive Gram points. We will classify the zeta values into two classes, '$+$' for positive zeta values ($zeta > 0$) and '$-$' for negative zeta values ($zeta \leqslant  0$). Thus, for example, the notation '$++$' stands for a Gram point which has a positive zeta value followed by a Gram point which has a positive zeta value, while the notation '$+-$' stands for a Gram point which has a positive zeta value followed by a Gram point which has a negative zeta value. Table~\ref{tab:pairraw} shows the counts for different pair configurations, for pairs beginning at odd Gram points (row 1) and for pairs beginning at even Gram points(row 2). Conjecture \ref{antisymmetry} predicts that the count for a configuration from the odd Gram point row in the table  will match the count for the mirror configuration in the even Gram point row of the table ( e.g., the count for '$++$' from the distribution for odd Gram points will match count for '$--$' from the distribution for even Gram points). Table~\ref{tab:pairtest} tests this prediction. The agreement is good, within the expected statistical variations. 

\section{\label{conclusions}Conclusions}

In this work we presented a new  relation for the ratio of good to bad Gram points. We presented two conjectures regarding the distribution of zeta function values at Gram points. The conjectures refer to the even-odd Gram point antisymmetry and the forward-backward symmetry (``time reversal" symmetry, if we think of the zeta values at Gram points as being a time series). After briefly describing the theory of the Riemann zeta function and the numerical evaluation, we presented statistics of the zeta function values. The statistics provide validation for the two conjectures.

implications and future work.


\begin{thebibliography} {}

\bibitem{Sarnak 2005} Peter Sarnak,
``Problems of the Millennium: The Riemann Hypothesis (2004)", The Proceedings of the Clay Mathematics Institute 2004, Budapest, Hungary,
\url{http://www.claymath.org/sites/default/files/sarnak_rh_0.pdf}, (2005)

\bibitem{Gram 1903} J. P. Gram, 
``Sur les Zeros de la Fonction  $\zeta ( s )$  de Riemann",
{\it Acta Math.} {\bf27}(1903), 289-304.

\bibitem{Odlyzko 1992}  A. Odlyzko,
``The $10^{20}$-th Zero of the Riemann Zeta
Function and 175 Million of its Neighbors", report,
\url{http://www.dtc.umn.edu/~odlyzko/unpublished/zeta.10to20.1992.pdf}, (1992)


\bibitem {Riemann(1858)} B. Riemann, ``\"{U}ber die Anzahl der Primzahlen uter
Einer Gegebenen Gr\"{o}be,'' {\it Montasb. der Berliner Akad.}, (1858),
671-680.

\bibitem {Riemann 1892} B. Riemann, ``Gesammelte Werke'', Teubner, Leipzig, (1892).

\bibitem {Titchmarsh 1986} E. Titchmarsh, ``The Theory of the Riemann Zeta
Function,'' Oxford University Press, Second Edition, (1986).

\bibitem {Edwards(1974)} H. M. Edwards, ``Riemann's Zeta Function,'' 
Academic Press,  (1974).

\bibitem{os6} O. Shanker, 
``Generalised Zeta Functions and Self-Similarity of Zero Distributions",
{\it J.  Phys. A} {\bf39}(2006), 13983-13997.

\bibitem {Matiyasevich} Y. Matiyasevich, 
``An artless method for calculating approximate values of
zeros of Riemann zeta function",
Web report, \url{http://logic.pdmi.ras.ru/~yumat/
personaljournal/artlessmethod/artlessmethodtexts.php}, (2013)

\bibitem{hiary} G. A. Hiary,
``METHODS TO COMPUTE THE RIEMANN ZETA
FUNCTION", arxiv.org, math.NT, 0711.5005v4, (2011).

\bibitem{gourdon} Xavier Gourdon,
``The $10^{13}$ first zeros of the Riemann Zeta function,
and zeros computation at very large height", report,
\url{http://numbers.computation.free.fr/Constants/Miscellaneous/zetazeros1e13-1e24.pdf}, (2004)

\bibitem {Odlyzko(1989)} A. Odlyzko, ``The $10^{20}$-th Zero of the Riemann Zeta
Function and 70 Million of its Neighbors'' (preprint), A.T.T., (1989).

\bibitem{hiary 2010} G. A. Hiary,
``An amortized-complexity method to compute the Riemann zeta function", report,
\url{https://people.math.osu.edu/hiary.1/amortized.html}, (2010).


\end{thebibliography} 

\end{document} 
