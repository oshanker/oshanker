%-----------------------------------------------------------------------
% 
%-----------------------------------------------------------------------
%
%     
%
%
%%%%%%%%%%%%%%%%%%%%%%%%%%%%%%%%%%%%%%%%%%%%%%%%%%%%%%%%%%%%%%%%%%%%%%%%


\documentclass[twoside]{article}
\usepackage{amsmath,amsthm,amssymb,verbatim}

%     If your article includes graphics, uncomment this command.
\usepackage{graphicx}

%     If the article includes commutative diagrams, ...
%\usepackage[cmtip,all]{xy}

\usepackage{url}

\usepackage{fancyhdr}
\pagestyle{fancy}

\def\blfootnote{\xdef\@thefnmark{}\@footnotetext} 
\long\def\symbolfootnote[#1]#2{\begingroup%
\def\thefootnote{\fnsymbol{footnote}}\footnote[#1]{#2}\endgroup} 

	\addtolength{\oddsidemargin}{1cm}
	\addtolength{\evensidemargin}{-1cm}

\setcounter{page}{1}

\begin{document}

%     If you need symbols beyond the basic set, uncomment this command.
%\usepackage{amssymb}


\newtheorem{theorem}{Theorem}[section]
\newtheorem{lemma}[theorem]{Lemma}

\theoremstyle{definition}
\newtheorem{definition}[theorem]{Definition}
\newtheorem{example}[theorem]{Example}
\newtheorem{xca}[theorem]{Exercise}

\theoremstyle{remark}
\newtheorem{remark}[theorem]{Remark}

\numberwithin{equation}{section}


\date{}
\lhead[]{}
\chead[\underline{Deep Learning for Riemann Zeta}]{\it{O. Shanker}}
\rhead[]{}

% \title[short text for running head]{full title}
\title{\bf{Deep Learning for Riemann Zeta Function: Large Values and Karatsuba problem}}

\maketitle


%    author one information
% \author[short version for running head]{name for top of paper}
\author{{\textbf{O. Shanker}},}
\thanks{ Mountain View, CA 94041, U. S. A. email: oshanker@gmail.com}

\thispagestyle{fancy}

%    Abstract is required.
\begin{abstract}
The study of large values of the Riemann zeta function on the critical axis is a topic of mathematical
interest.  One such mathematical topic is the Karatsuba problem. It will be useful to have empirical results on the distribution of large values of the zeta function, to give insights 
into the theoretical studies.  Since evaluating the Riemann zeta function at large heights  is a non-trivial task, requiring much computer time 
(and some knowledge of special techniques to find the roots), we apply
machine learning to the problem.

\end{abstract}
{\textbf {Keywords}:} Circular Unitary Ensemble, Riemann zeta, Value Distribution, Symmetry 
{\textbf {Mathematics Subject Classification (MSC)}:} 11M06, 11-04.


\symbolfootnote[0]{*}


\section{Introduction}

 There have been many applications of machine learning in mathematics, including 
the application of Neural networks to the Riemann zeta function~\cite{osneural}.
Here we study its use to provide insights into  the Karatsuba problem see Ref~\cite{K5, Kor1}.


\section{\label{sec2}Materials and Methods}

\begin{figure*}
\centering
\includegraphics[width=0.8\textwidth]{1.png}
\caption[]{ 
  Distribution of $\zeta_{max}$. 
  }
\vspace{1mm}
\label{z1}
\end{figure*}

\subsection{\label{seckaratsuba}Karatsuba Problem}

Among the many studies of large values of the Riemann zeta function on the critical axis, Karatsuba~\cite{K5} studied 
\begin{equation}
F(T; H)  \, = \, max_{|t-T| \le H} \zeta ( 0.5+it ) 
\label{eqRie}
\end{equation}
where $H$ is small compared to $T$. The ramifications of the Karatsuba problem are explained and
reviewed in, e.g.,  Korolov~\cite{Kor1}.  Fig.~\ref{z1} gives the distribution of $F(T; H)$ for $T=10^{12}, H=26$.

The freezing transition scenario occurs in the statistical mechanics of 1/f-noise random energy models. Ref~\cite{FK} argue that it  also governs the value distribution of the maximum of the modulus of the characteristic polynomials  of large N × N (circular unitary ensemble, CUE) random unitary  matrices for large N.  Given the relation of CUE matrices to the Riemann zeta function~\cite{oscue}, the arguments have implications to the topic we are studying.


\subsection{\label{secwhy}Why Machine Learning?}

At large heights evaluating the Riemann zeta function  is a non-trivial task, requiring much computer time 
(and some knowledge of special techniques to find the roots).  It would be useful to apply
machine learning
as a guide to identify the T values where we can expect to see the behaviour of interest.
Ref~\cite{osneural} found that the behavior of the zeta function at Gram points 
is a good starting point to extract features for use in prediction. 
Ref.~\cite{Shanker 2018a} further showed that Gram points have interesting properties 
which distinguish them from random points on the critical line, supports the conclusion. 
Related  results can be found in Ref.~\cite{os6, Shanker 2018b,Shanker 2020}.

We use TensorFlow~\cite{FrancoisChollet 2021}.

\begin{figure*}
\centering
\includegraphics[width=0.8\textwidth]{2.png}
\caption[]{ 
  LSTM model. 
  }
\vspace{1mm}
\label{z2}
\end{figure*}

\begin{figure*}
\centering
\includegraphics[width=0.8\textwidth]{3.png}
\caption[]{ 
  Training and Validation Mean Absolute Deviations.
  }
\vspace{1mm}
\label{z3}
\end{figure*}

\begin{figure*}
\centering
\includegraphics[width=0.8\textwidth]{4.png}
\caption[]{ 
  Comparision of model prediction with actual $\zeta_{max}$. 
  }
\vspace{1mm}
\label{z4}
\end{figure*}

\subsection{\label{seccalc}Calculating the sample values}

At large heights evaluating the Riemann zeta function  is a non-trivial task, requiring much computer time 

\section{\label{sec3}Results}

\subsection{\label{sec3.1} Choosing the Model}

Since the Riemann zeta studies were done at a
height $T = 10^{12}$, we use the well-known correspondence $N \thicksim \ln(T/2\pi)$ where $N$ 
is the size of the unitary matrix we should consider. Thus, we chose $N = 26$ for our
study.  We studied $500000$ gram intervals $U$ to generate the distributions.

have $N$ distinct CUE generalized Gram points spaced uniformly
around the unit circle.
The definition for the probability distribution is analogous to the Riemann zeta definition.

 (see ~\ref{z2},~\ref{z3},~\ref{z4}) 

\begin{table}
\centering \(\begin{array}{cccccccccccc}

\hline
learning     &momentum  &epochs  \\
rate    &  &  \\
\hline
0.001 &0.895  & 15  \\
\hline
\end{array}\)
\caption{LSTM Model parameters (optimizer=tf.keras.optimizers.RMSprop)}
\label{tab:mean12}
\end{table}


\subsection{\label{relation}Training,  Model predictions}

For the 

\section{\label{conclusions}Conclusions}

We the Riemann zeta function.

\section*{Acknowledgments and Funding Statement}

 The study was done as an independent researcher. There was no
external funding.

\section*{Ethical Compliance}

 No procedures were performed  involving human participants in the study.

\section*{Data Availability Statement}

The computer programs used during the current study are
available from the corresponding author on reasonable request.

\section*{Conflict of Interest declaration} 

The authors declare that they have no affiliations with or involvement in any organization 
or entity with any financial interest in the subject matter or materials discussed 
in this manuscript.


\bibliographystyle{amsplain}
\begin{thebibliography}{10}

\bibitem{osneural} O. Shanker, ``Neural Network prediction of Riemann zeta zeros''
{\it Advanced Modeling and Optimization}, {\bf 14}, 717-728, (2012). 

\bibitem{K5} A. A. Karatsuba, 
``Zero multiplicity and lower bound estimates of $|\zeta(s)|$",
{\it Funct.  Approx. Comment. Math.} {\bf35}(2006), 195–207

\bibitem{Kor1} Maxim A. Korolev, 
``On large values of the Riemann zeta-function on short segments of the critical line",
{\it Acta Arithmetica} {\bf166}(2014), 349–390

\bibitem{FK} Y. V. Fyodorov and J. P. Keating,
``Freezing transition and extreme values random matrix theory, $\zeta(1/2 + it)$, and disordered landscapes",
{\it Phil. Trans. R. Soc. A} {\bf372}(2014),  20120503

\bibitem{oscue} O. Shanker, 
``Random Matrix Theory explanation for Riemann Zeta Value Distribution Symmetry''
 report,
\url{https://tinyurl.com/ywhy4jsy}, 
(2022). 

\bibitem{Shanker 2018a} O. Shanker, 
``Good to Bad Gram Point Ratio For Riemann Zeta Function",
{\it Experimental Mathematics} {\bf 30}, 76-85,
\url{tinyurl.com/mwd5uwc5}(2021)

\bibitem{os6} O. Shanker, 
``Generalised Zeta Functions and Self-Similarity of Zero Distributions",
{\it J.  Phys. A} {\bf39}(2006), 13983-13997

\bibitem{Shanker 2018b} O. Shanker, 
``Symmetry properties of distribution of Riemann Zeta Function values on critical axis''
 report,
\url{tinyurl.com/47wj57b3}, 
(2018). 

\bibitem{Shanker 2020} O. Shanker, 
``Universality of Riemann Zeta Function value distribution on critical axis''
 report,
\url{tinyurl.com/yvbd2je6}, 
(2020). 

\bibitem {FrancoisChollet 2021} Francois Chollet, ``Deep Learning with Python''
Manning Publications,  (2021)




\end{thebibliography} 

\end{document}
