%-----------------------------------------------------------------------
% 
%-----------------------------------------------------------------------
%
%     
%
%
%%%%%%%%%%%%%%%%%%%%%%%%%%%%%%%%%%%%%%%%%%%%%%%%%%%%%%%%%%%%%%%%%%%%%%%%


\documentclass[twoside]{article}
\usepackage{amsmath,amsthm,amssymb,verbatim}

%     If your article includes graphics, uncomment this command.
\usepackage{graphicx}

%     If the article includes commutative diagrams, ...
%\usepackage[cmtip,all]{xy}

\usepackage{url}

\usepackage{fancyhdr}
\pagestyle{fancy}

\def\blfootnote{\xdef\@thefnmark{}\@footnotetext} 
\long\def\symbolfootnote[#1]#2{\begingroup%
\def\thefootnote{\fnsymbol{footnote}}\footnote[#1]{#2}\endgroup} 

	\addtolength{\oddsidemargin}{1cm}
	\addtolength{\evensidemargin}{-1cm}

\setcounter{page}{1}

\begin{document}

%     If you need symbols beyond the basic set, uncomment this command.
%\usepackage{amssymb}


\newtheorem{theorem}{Theorem}[section]
\newtheorem{lemma}[theorem]{Lemma}

\theoremstyle{definition}
\newtheorem{definition}[theorem]{Definition}
\newtheorem{example}[theorem]{Example}
\newtheorem{xca}[theorem]{Exercise}

\theoremstyle{remark}
\newtheorem{remark}[theorem]{Remark}

\numberwithin{equation}{section}


\date{}
\lhead[]{}
\chead[\underline{head}]{\it{O. Shanker}}
\rhead[]{}

% \title[short text for running head]{full title}
\title{\bf{Title}}

\maketitle


%    author one information
% \author[short version for running head]{name for top of paper}
\author{{\textbf{O. Shanker}},}
\thanks{ Mountain View, CA 94041, U. S. A. email: oshanker@gmail.com}

\thispagestyle{fancy}

%    Abstract is required.
\begin{abstract}
Abstract. 
texworks
--tex-option="--output-directory=/Users/uorugant/junk/oshanker for pdfLatex+.. "  in prefs
doesn't work.
\url{https://tex.stackexchange.com/questions/12417/texify-and-output-directory}

\end{abstract}
{\textbf {Keywords}:} Circular Unitary Ensemble, Riemann zeta, Value Distribution,  Large Values 
{\textbf {Mathematics Subject Classification (MSC)}:} 11M06, 11-04.


\symbolfootnote[0]{*}


\section{Introduction}

 There .


\section{\label{sec2}Materials and Methods}
Section Intro

\subsection{\label{seckaratsuba}Karatsuba Problem}

Among the many studies of large values of the Riemann zeta function on the critical axis, 
studied 
\begin{equation}
F(T; H)  \, = \, max_{|t-T| \le H} \zeta ( 0.5+it ) 
\label{eqRie}
\end{equation}
where $H$ is small compared to $T$. 

The freezing tra .  Figure reference Figure~~\ref{tab:ratioE28}.

\subsection{\label{secwhy}Why Machine Learning?}

At large 

\begin{figure*}
\includegraphics[width=1.0\textwidth]{typeIIratio.pdf}
\caption[]{ 
 Plot of  ratios in  Table~\ref{tab:ratioE28},  $t = 10^{28}$.
 }
\vspace{1mm}, 
\label{fig:ratioE28}
\end{figure*}


\begin{figure*}
\includegraphics[width=1.0\textwidth]{Round1.png}
\caption[]{ 
 Round 1.
 }
\vspace{1mm}, 
\label{fig:round1}
\end{figure*}


\begin{figure*}
\includegraphics[width=1.0\textwidth]{Round2.png}
\caption[]{ 
 Round 2.
 }
\vspace{1mm}, 
\label{fig:round2}
\end{figure*}


\begin{table}
\centering \(\begin{array}{cccccc}
\hline
Length~of  && Generalized &Gram&angle &\phi \\
Gram     &----&----&----&----&----\\
block  & -0.2\pi & -0.1\pi & 0.0\pi & 0.1\pi & 0.2\pi  \\
\hline
2 &2.269 &1.505 &0.999 &0.664 &0.441 \\
3 &3.625 &1.896 &0.999 &0.526 &0.275 \\
4 &5.589 &2.367 &1.001 &0.427 &0.178 \\
5 &8.850 &2.923 &1.012 &0.343 &0.115 \\
6 &14.373 &3.729 &1.009 &0.266 &0.071 \\
7 &23.962 &4.722 &0.975 &0.205 &0.041 \\
8 &51.791 &6.715 &0.984 &0.149 &0.018 \\
9 &116.632 &10.130 &0.975 &0.104 &0.009 \\
10 &361.615 &13.306 &0.949 &0.058 &0.005 \\
11 &1397.000 &34.533 &1.028 &0.041 &0.001 \\
\hline
\end{array}\)
\caption{Sharp transition in $Type~II/Type~I$ Gram block ratio.
The statistics are from $10$ million Gram intervals at $t=10^{28}$.}
\label{tab:ratioE28}
\end{table}



\subsection{\label{sec3.1} Choosing the Model}
The data we are s

\subsection{\label{relation}Training,  Model predictions}

shows the model prediction versus the actual value for some points.

Epoch: 1, Train loss: 1.131, Train acc: 0.675, Val loss: 0.567, Val acc: 0.716 Epoch time = 2.153s

Epoch: 1, Train loss: 0.264, Train acc: 0.914, Val loss: 0.011, Val acc: 0.998 Epoch time = 2.918s


\section{\label{conclusions}Conclusions}
We 

\section*{Acknowledgments and Funding Statement}

 The study was done as an independent researcher. There was no
external funding.



\bibliographystyle{amsplain}
\begin{thebibliography}{10}

\bibitem{osneural} O. Shanker, ``Neural Network prediction of Riemann zeta zeros''
{\it Advanced Modeling and Optimization}, {\bf 14}, 717-728, (2012), \url{tinyurl.com/4scve3nj}.


\bibitem{oscue} O. Shanker, 
``Random Matrix Theory explanation for Riemann Zeta Value Distribution Symmetry''
 report,
\url{https://tinyurl.com/ywhy4jsy}, 
(2022). 


\bibitem{Shanker 2018a} O. Shanker, 
``Good to Bad Gram Point Ratio For Riemann Zeta Function",
{\it Experimental Mathematics} {\bf 30}, 76-85,
\url{tinyurl.com/mwd5uwc5}(2021)

\bibitem{os6} O. Shanker, 
``Generalised Zeta Functions and Self-Similarity of Zero Distributions",
{\it J.  Phys. A} {\bf39}(2006), 13983-13997

\bibitem{Shanker 2018b} O. Shanker, 
``Symmetry properties of distribution of Riemann Zeta Function values on critical axis''
 report,
\url{tinyurl.com/47wj57b3}, 
(2018). 

\bibitem{Shanker 2020} O. Shanker, 
``Universality of Riemann Zeta Function value distribution on critical axis''
 report,
\url{tinyurl.com/yvbd2je6}, 
(2020). 




\end{thebibliography} 

\end{document}
