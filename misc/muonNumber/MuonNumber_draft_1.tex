%-----------------------------------------------------------------------
% 
%-----------------------------------------------------------------------
%
%     
%
%
%%%%%%%%%%%%%%%%%%%%%%%%%%%%%%%%%%%%%%%%%%%%%%%%%%%%%%%%%%%%%%%%%%%%%%%%


\documentclass[twoside]{article}
\usepackage{amsmath,amsthm,amssymb,verbatim}

%     If your article includes graphics, uncomment this command.
\usepackage{graphicx}

%     If the article includes commutative diagrams, ...
%\usepackage[cmtip,all]{xy}

\usepackage{url}

\usepackage{fancyhdr}
\pagestyle{fancy}

\def\blfootnote{\xdef\@thefnmark{}\@footnotetext} 
\long\def\symbolfootnote[#1]#2{\begingroup%
\def\thefootnote{\fnsymbol{footnote}}\footnote[#1]{#2}\endgroup} 

	\addtolength{\oddsidemargin}{1cm}
	\addtolength{\evensidemargin}{-1cm}

\setcounter{page}{1}

\begin{document}

%     If you need symbols beyond the basic set, uncomment this command.
%\usepackage{amssymb}


\newtheorem{theorem}{Theorem}[section]
\newtheorem{lemma}[theorem]{Lemma}

\theoremstyle{definition}
\newtheorem{definition}[theorem]{Definition}
\newtheorem{example}[theorem]{Example}
\newtheorem{xca}[theorem]{Exercise}

\theoremstyle{remark}
\newtheorem{remark}[theorem]{Remark}

\numberwithin{equation}{section}


\date{March 1981}
\lhead[]{}
\chead[\underline{Present Status of Muon Number}]{\it{O. Shanker}}
\rhead[]{}

% \title[short text for running head]{full title}
\title{\bf{Present Status of Muon Number}}

\maketitle


%    author one information
% \author[short version for running head]{name for top of paper}
\author{\textbf{O. Shanker}},
\thanks{ TRIUMF, 4004 Wesbrook Mall, Vancouver, B.C., Canada VET 243 }

\thispagestyle{fancy}

%    Abstract is required.
\begin{abstract}
(Note: Resurrected from an  old work of mine, using modern image processing technology.
The work was done in 1981, so this is just for reference! In those days I had done work
on Leptoquarks and on Supersymmetric particles, but that work did not make it into this
writeup. I have not made any changes to the writeup.)

This work surveys the present experimental and theoretical status
of muon number. The development of our ideas regarding flavour numbers
is discussed using muon number as an example.
The role of muon number
in different models that are currently being studied is investigated.
The phenomenological and experimental status of different muon number
violating processes is reviewed. The predictions of different models
for muon number violating processes is discussed.

\end{abstract}
{\textbf {Keywords}:} muon number violation


\symbolfootnote[0]{TRIUMF Report TRI-PP-81-10 }



\section*{Contents}


\begin{enumerate}
\item Introduction

\item  Leptonic processes


A. $\mu$ -> ey

B.  $\mu$ -> eee

C. w-~>eyy

D. Je + eu

E. Neutrino processes

\item  Semi-leptonic processes

A. yu~exxx conversion

B. u~et conversion

C. Kaon decays

D. No-neutrino double beta decay

\item  Model predictions

A. Horizontal gauge models

B. Hypercolour models

C. Grand unified theories

D. Neutrino mass

E. Higgs models

\item Conclusions
\end{enumerate}

\begin{itemize}
\item Acknowledgments


\item End Matter

\item Figures
\item References
\end{itemize}

\section{Introduction}
The weak interactions have shown a notorious disrespect for our
ideas of what the symmetries of nature ought to be. The violation of
parity and the even more startling violation of CP provide examples of
this notoriety. In the field of flavour numbers the perversity of the
weak interactions manifests itself from the opposite direction, namely,
nature presents physicists with a lot of symmetries which have no good
reason for existing. For example, since the time the muon was found to
behave like a heavy electron physicists have been debating over the question of why it exists. The symmetries of nature which account for its
existence and the existence of other flavour numbers lke electron
number, strangeness, charm etc. are still not completely understood,
and seem to be fortuitous symmetries. A number of papers dealing with
possible small violations of muon number, baryon number, lepton number,
etc. continue to appear in the literature. The motivation for possible
violations of baryon number and lepton number is provided by Grand
Unified theories, which are reviewed by Langacker (1980). In this paper
I study the role of muon number in present ideas of particle physics.
A pre-gauge theory review of muon number can be found In Frankel, 1974 and
Pontecorvo, 1967. Cheng and Li (1977a) and Weinberg (1977a and 1978) give a
large number of references to the early literature on muon number violation
after the advent of spontaneously broken gauge theories. In the present work
the development of our ideas regarding flavour numbers is briefly reviewed by
looking at muon number as an example. | then look at the role of muon num-
ber in different extensions of the Glashow-Weinberg-Salam model that are currently of interest. After a review of the phenomenological and experimental
status of different muon number violating processes, I conclude by discussing which of these processes are favoured by different gauge models.


The idea of flavour numbers is in essence a description of the ultimate building blocks of nature. There is a close similarity between the
basic elements of the ancient Greeks and Indians or the immutable chemical elements of Dalton and flavour numbers. However, the latter term
is used only in relation to the modern concepts regarding the ultimate
structure of matter. The discoveries at the end of the nineteenth century (discoveries of the electron, of the transmutability of elements,
etc.) indicated that the basic building blocks of nature were far fewer
than the ninety or so elements discovered till then. In the early decades
of this century two fundamental building blocks, the proton and electron,
were believed to be the constituents of all matter. With the realisation
that nuclear beta decay involved a new massless particle, the neutrino,
a study of its nature was taken up. Racah (1937) and Pontecorvo (1950)
suggested experiments to see if the neutrino and its antiparticle were
identical. The radlochemical experiments of Davis (1955) established the
separate existence of the neutrino and antineutrino. It now seemed as
if al) interactions in nature conserved two quantities, baryon number
and lepton number. This simple situation did not last long and the
discovery of the neutron, muon, pion and strange particles in the first
half of this century showed that nature was more complex. Two new flavours, strangeness (Gell-Mann, 1953; Nishijima, 1955) and muon number,
were introduced within a short time of each other. The introduction of
these new flavours was phenomenological, i.e., it merely served to explain the absence of certain particle reactions and did not lead to any
new insights into the details of weak interactions. Since that time our
understanding of flavour numbers and their conservation schemes has developed quite a lot, but we still do not understand why so many flavours
exist in nature (in fact, many new flavours have been discovered since
then and we have no idea how many more flavours may exist). The brief
historical sketch of the muon that | now present serves to illustrate
the puzzlement that physicists felt, and still feel, about the existence
of so many flavours.

The muon was discovered in cosmic rays by Anderson and Neddermeyer
(1938) and by Street and Stevenson (1937). At first it was confused with
the particle predicted by Yukawa as mediating strong interactions. However, its stability in nuclear environments showed that it did not have
a strong interaction with nucleons and hence could not be the particle
predicted by Yukawa. In the forties it was established that the muon was
very similar to the electron and seemed to differ from it only in mass.
Hence physicists expected it to decay into the electron without accompanying neutrinos, in addition to its decay modes involving final state neutrinos. However, searches of increasing accuracy in the fifties and sixties for processes like yu + ey, ut + ete”e~ and ne conversion failed to
find them. The theoretical predictions for the rates of such processes
were beset with problems of divergent integrals and artificial cutoffs
because these decays proceeded in second order in the old non-renormalisable
models of weak interactions [current X current model or the old Inter-
mediate vector boson model with only one type of neutrino, reviewed in
Frankel (1974)]. These problems made it difficult to decide if the experimental upper limits suggested modifications in the theory of weak interactions. There seemed to be growing evidence for the existence of two
types of neutrinos and for a new quantum number, "muon number''. This
idea gained support with the introduction of intermediate vector boson
theories of the weak interaction (Feinberg, 1958; Schwinger, 1957).


Experimental tests for the new quantum number were discussed by 
Pontecorvo (1959), Schwartz (1960) and Lee and Yang (1960). The need for 
introducing a new flavour was definitely established by the experiments ter of the weak Interactions plays an important role in suppressing un-
of Danby et al. (1962), who showed that when neutrinos produced from wanted reactions like p~et conversion and K* + pte
when the mass of
the decay of pions into muons Interacted with nuclei, they produced the neutrino is zero all reactions forbidden by the additive scheme are
muons but not electrons. The rate for production of electrons and also forbidden in this scheme. The relation between the suppression of
muons in these processes is unambiguously calculable even in the old unwanted processes and the V-A character of the weak currents is nicely
models of weak interactions and a conflict between theory and experiment Illustrated by the work of Primakoff and Rosen (1972) who Introduce V+A
clearly emerged. Thus, it became necessary to introduce a new type of flavour. currents and calculate the rates for anamolous processes. A third scheme
The phenomenological nature of the old theories Is well illustrated is the multiplicative scheme introduced by Feinberg and Weinberg (1961a).
by the many different types of lepton number schemes that were proposed. In this scheme there is an additively conserved lepton number L and a
The only way of choosing among them was to look for processes that were multiplicatively conserved lepton number Lp. All reactions satisfy
permitted in some schemes but not In others. The lepton number schemes EW nie1a1= ED final and (nlp) inteiar=(rtp) final - wT, eT, wy, and ve
did not predict rates for the "exotic" processes. The need to probe
have L=1 and their antiparticles have L=-1. et, e7, ve and Ve have Lp=1 and ut, y=, vy and vy have Lp=-1.
the validity of the different schemes was one reason for the experimental interest in lepton number violating processes in the seventies.  This scheme is less restrictive than
 the previous ones and allows reactions like $\mu^{+}+e^{-} -> \mu^{-}+e^{+}$ , ut -> 3e ov
Three lepton number conservation schemes have become commonly known etc. The limits on the phenomenological coupling constants are comparable
(Frankel, 1974; Pontecorvo, 1967). The additive lepton number scheme to the weak interaction coupling constant Gf If one uses phenomenological
(Nishijima, 1957; Bludman, 1958; Schwinger, 1957) is the most widely known currents for these processes. It is not straightforward to extend the
and also the most restrictive. In this scheme there are two lepton Konopinski-Mahmoud scheme or the Feinberg-Weinberg scheme to take into
numbers, Le and L,, with e” and ve having Le=l and L,=0, and u~ and Vu account the t lepton discovered in 1975 (Perl et al., 1975).
having Ly=1 and Le=0. The antiparticles have the opposite sign for The advent of spontaneously broken renormalisable gauge theories of
these quantum numbers. These lepton numbers are additively conserved, weak and electromagnetic interactions (Abers and Lee, 1973; Taylor, 1976)
i.e., Tie and Zu, do not change in any interaction. Another lepton led to constraints on the kinds of lepton number schemes allowed. |
number scheme, actually the first lepton number scheme, is due to describe these theories now in some detail because of their importance
Konopinski and Mahmoud (1953). In this scheme there is only one lepton and because they form the framework for the rest of the discussion.
number L* which is additively conserved. y= and e* have L*=1, there is only one neutrino which can exist in both negative and positive helicity. states and has L=1, and the antiparticles have L*=-1. The V-A charac~

Readers familiar with these theories may want to skip this paragraph
R -6 -
and the next. Gauge theories belong to the Lagrangian approach to par-
ticle physics. In this approach particles are described by local fields
(wave functions) and particle interactions are derived according to cer-
tain prescriptions from a Lagrangian which is a function of the particle
fields. The prescriptions include an arbitrary subtraction of certain
infinite quantities to give finite values, a procedure called renormal-
isation. Field theories in which the renormalisation procedure is well
defined are called renormalisable theories. Renormalisable theories are
very important because they are the only type of theories which seem to
be able to make non-trivial predictions for particle interactions. In
Lagrangian field theories there is a close relation between conserved
quantities and symmetries of the Lagrangian. For every conserved quan-
tity there exists a continuous transformation which leaves the Lagrangian
invariant. Conversely, for every continuous transformation which leaves
the Lagrangian invariant one can find a conserved quantity. This is
known as Noether's theorem and is reviewed in (Hill, 1957). In general
the transformation is a global transformation, i.e., the wave functions
in the Lagrangian mix among themselves in the same way at every point of
space and time. In the gauge theories particles are assigned to repre-
sentations of some gauge group. The Lagrangian remains unchanged when
the wave functions of the particles transform among themselves under a
gauge transformation according to the group representations to which the
particles are assigned. To extend this symmetry to local gauge trans-
formations (i.e., the gauge group parameters are functions of space-time
coordinates) one has to introduce massless gauge bosons. These gauge
bosons interact with the fermions in such a way as to make the Lagrangian
invariant under local gauge transformations. Local gauge theories are

appealing because particle interactions are not arbitrary but are fixed by the
choice of the gauge group and the particle representations. In addition,
when calculating S-matrix elements one can use the gauge symmetries to
prove that these theories are renormalisable. At present local gauge
theories seem to provide a correct description of particle interactions.
Since the only massless gauge boson known to us is the photon one has
to find a mechanism for giving masses to the other gauge bosons
occurring in the theories. A mechanism must also exist to break the
mass degeneracy between fermions belonging to the same representation
of the gauge group. Both can be achieved by introducing scalar bosons
(Higgs bosons) which interact in such a way that the theory is spontan-
eously broken, i.e,, the ground state of the theory does not have the
symmetry of the Lagrangian. An example of spontaneous symmetry break-
ing in solid state physics is provided by a ferromagnetic crystal. In a
ferromagnetic crystal the spins of the atoms have a tendency to align
themselves in one direction. A priori, there are many equivalent direc-
tions in which they can align themselves. However, if a few spins get
aligned in one of these directions (due to thermal fluctuations, a magnetic
field which is momentarily applied, or for any other reason), the other
spins will follow suit very quickly and the crystal does not fully repre-
sent the symmetry of the spin-spin interaction. It is this reduction in
symmetry which sometimes inhibits phase transitions in ultra-pure systems.
The system must choose one of several allowed configurations, and in conden-
sed matter physics this choice is often made by the impurities (nuclea-
tion centres for condensation, the magnetic field in the above example,
and so on).
For the sake of completeness | should mention that not all local
gauge theories are renormalisable., In the proof of renormalisability
-8 -
one uses certain identities, called the Ward identities, which are de-~
rived from gauge invariance. Sometimes the derivation of these identi-
ties (and hence the proof of renormalisability) cannot be carried through
if the Feynman diagrams of the theory have triangle anomalies (Bardeen,
1974). (Feynman diagrams are diagrams which tel) us how to calculate
physical cross sections, distributions, etc. from the Lagrangian. |
know of no simple way to describe triangle anomalies.) Stated another
way, triangle anomalies imply that the gauge invariance of the Lagran-
gian is not maintained by the Feynman diagrams of the theory. This loss
of symmetry comes about because the Feynman diagrams have to be renormal-
ised, and no renormalisation procedure can be found which maintains gauge
invariance. The triangle anomalies do not always spoil the renormalis-
ability of the theory. If the reader will permit a non-serious digression
| would like to mention another triangle anomaly discovered earlier by
psychologists. The psychologists' triangle anomaly is illustrated in Fig. 1.
A cursory examination shows that two shades of white have been used in the figure.
However, a closer inspection reveals that the white triangle is actually
the same shade as the background whitel Thus, unlike the particle
physics triangle anomaly where a symmetry was lost in going from the
Lagrangian to the Feynman diagrams, this triangle anomaly reveals a hid-
den symmetry. A more serious (and technical) explanation of the particle
physics triangle anomaly can be found in (Adler, 1970; Jackiw, 1970).
In the spontaneously broken gauge theories there is a close connec-
tion between the assignment of particles to group representations, and
lepton number schemes. The gauge bosons induce transformations between
particles belonging to the same group multiplet. Thus, one can introduce
a "flavour number" to each multiplet and the gauge interactions will
conserve this flavour number. For example, In the standard model (Glashow,

1961; Weinberg, 1967; Salam and Ward, 1964) the electron and the muon belong to
different SU(2),eak doublets and the additive lepton number conservation
scheme is realised by the model. In a slightly different model proposed
by Derman (1979) some new interactions exist In addition to the usual
weak and electromagnetic interactions. These new interactions result in
small violations of the usual additive muon number and other lepton
numbers. However, one can still define multiplicatively conserved lepton
numbers and the model realises a multiplicative lepton number conservation
scheme. Weinberg (1972) studied a gauge model where the Konopinski-
Mahmoud lepton number scheme is realised. Thus, in spontaneously broken
gauge theories flavour numbers are not arbitrary but have to be explained
in terms of particle representations. Assigning fermions to different
group representations is necessary but not sufficient to ensure flavour con-
servation, because eigenstates of the weak interactions are not eigenstates of
the mass matrix. Due to spontaneous symmetry breaking, states with the same
charge and helicity will mix and the weak Interactions will violate fla-
vour number, with the order of magnitude of the flavour violation depen-
ding on the mixing angles. An example is the Cabibbo angle and non-
conservation of strangeness in weak interactions. Flavour violation is
said to be suppressed naturally if the suppression is due to the repre-
sentation content of the theory and not due to arbitrarily small para-
meters. For example, flavour non-conservation is suppressed naturally
in the standard model because the GIM mechanism {Glashow et al., 1976)
explains the very small strength of the observed strangeness-changing
neutral currents in terms of the gauge multiplet assignments of the
quarks. Flavour number violation In gauge theories can occur due to
lepton mixing and a non-diagonal mass matrix (Cheng and Li, 1977a;
Kobayashi and Maskawa, 1973), due to flavour changing gauge bosons
. - 10 ~
(Deshpande, 1979; Mannheim, 1978; Maehara and Yanagida, 1979), and due
to flavour number violating Higgs particle couplings (Bjorken and
Weinberg, 1977; McWilliams and Li, 1980a). Flavour violation could also
occur due to anomalies ('t Hooft, 1976). However, the effect is very
small and has no practical consequences.
Thus, we see that spontaneously broken gauge theories provide the
best framework to discuss lepton number conservation schemes and poss-
ible deviations from these schemes. It is very easy to incorporate a
small amount of lepton number violation in gauge theories as shown by
the avalanche of gauge theories which followed the rumour in 1975 that
the p + ey process had been observed (the rumour turned out to be wrong).
If muon number is to be conserved exactly in a gauge theory, then some
restrictive conditions must be satisfied, namely, the muon and electron
must occur in separate multiplets, some mechanism must be found for en-
suring that the lepton mixing does not result in violation of muon num-
ber, and the Higgs sector must also be chosen to conserve muon number.
In the standard SU(2) x U(1) model with six quarks and six leptons all
these conditions are satisfied. The muon and electron occur in different
doublets, the absence of right handed neutrinos ensures their masslessness
and hence all lepton mixing angles can be made zero by properly defining
the muon and electron neutrino states (this can be done because of the de-
generacy in their masses), and the presence of only one Higgs doublet en-
sures that the Higgs-lepton interactions conserve muon number even after
the lepton mass matrix is diagonalised. While the Glashow-Weinberg-Salam
model predicts exact conservation of muon number, any modification of
this model usually results in a small amount of muon number violation.
The Glashow-Weinberg-Salam (GWS) model agrees very well with
essentially all present experimental results. Hence one may ask if

there Is any reason to expect muon number violation. The answer is that
violation of muon number is not unreasonable, firstly, because the ex-
perimental data do not severely constrain the Higgs sector in the GWS
model and hence the violation of muon number through the Higgs sector is
not ruled out, and secondly, because there are many questions that the
GWS theory doesn't answer, for example: why are right-handed neutrinos
absent? Why does nature repeat herself, with the electron family, the
muon family, the t~lepton family and maybe more (Harari, 1979)? It
would be nice to have a theory which explained such facts instead of
taking them as assumptions, and one has to go beyond the GWS model for
this. Any theory which attempts to answer these questions also generally
predicts a small violation of muon number.
One popular reason to consider modifications of the elementary GWS
model is the attempt to calculate quark mixing angles in terms of their
masses. In these attempts additional discrete or continuous symmetries
involving transformations among the different fermion generations are
imposed on the particles in the theory. The additional symmetries can
be chosen so that natural flavour conservation (NFC) is imposed on the
theory. However, general theorems exist (Kang and Rothman, 1980; Segre
and Weldon, 1979) which show that if one wishes to have calculable quark
mixing angles one must also accept flavour violation at some level. No
compelling model exists which predicts mixing angle relations.
Another reason to consider modification of the GWS model is to gen-
erate CP violation spontaneously (Lee, 1973 and 1974). This possibility
is interesting because it explains naturally why we can presently
measure CP violation only in the Ko-KO system. G.C. Branco (1980) has
shown that if NFC is imposed on the theory the Higgs exchanges provide
the only source of CP violation. In this class of models the neutron
 = 13 -
electric dipole moment can be relatively large. The CP violating kaon hierarchy problem in grand unified theories (the gauge hierarchy pro-
decay parameter e'/c may also be very large (Deshpande, 1981; Sanda, 1981). If blem is the problem of getting naturally two levels of spontaneous
natural flavour conservation Is not imposed one can get spontaneous CP symmetry breaking at vastly different energies, one at the grand unifi-
violation in the Weinberg-Salam model with two Higgs doublets. Lahanas cation energy of about 10!5 GeV and the other at present energies of a
and Vayonakis (1979) have considered such a model and found that the few hundred GeV.) In dynamically broken theories only fermions and
Higgs masses are of the order of a TeV. Their model has an interesting gauge bosons are introduced. All interactions are determined by the
feature, namely, the Higgs contribution to the $K_{L}-K_{S}$ mass difference is local gauge principle. The role of the elementary Higgs particles of
proportional to Higgs particle mass differences. the usual theories is now played by bound fermion-anti~fermion compo-
SU(2) x SU(2)R x U(1) models have also been considered for gener- sites. These theories are potentially attractive because in principle
ating spontaneous CP violation and to have a theory which is invariant
they have very few arbitrary parameters. No successful theory has been
)
under parity at high energies (Pati and Salam, 1974; Mohapatra and Pati, constructed as yet, but much work is being done on the problem.’ These
1975; Senjanovic and Mohapatra, 1975). In the left-right symmetric theories predict flavour changing neutral currents leading to flavour
models it is possible to identify the U(1) generator with {(B-L), where violating rates at levels not far below present experimental upper lim-
B is baryon number and L is lepton number (Marshak and Mohapatra, 1980a; its (Eichten and Lane, 1980; Ellis et al., 1980; Dimopoulos and Ellis, 1980).
Mohapatra and Marshak, 1980a), and hence to study violation of (B-L). It Horizontal gauge models, in which neutral "horizontal' gauge bosons
is also possible to relate the small mass of the neutrinos to the large * mediate transitions between members of different generations, have been
masses picked by the gauge bosons of SU(2)g (Mohapatra and Senjanovic, considered by many authors (Maehara and Yanagida, 1979; Wilczek and Zee,
1980b), and thus to the scale at which parity Is spontaneously broken. 1979a; Ramond, 1979; Davidson and Wali, 1980; Chikashige et al., 1980; Davidson
These models have heavy right-handed Majorana neutrinos (with mass of et al., 19793, and 1979b; Barr and Zee, 1978; Kane and Thum, 1980; Cahn and
order 100 GeV). Flavour violation is a very general feature of left- Harari, 1980; Shanker, 1980 and 1981; Sato, 1980; Ong, 1980; Ecker et al.,
right symmetric models (Chaniowitz et al., 1977). This has been studied 1981; Chakrabarti et al., 1980, and references therein, and references in
by Marshak and Mohapatra, 1980a, Mohapatra and Marshak, 1980a, and Langacker, 1980). In such models the repetition of generations is ex-
Mohapatra and Senjanovic, 1980b. They study the neutrinoless double plained naturally, and the families are identified by the representations
beta decay and the u + ey process, but do not make a complete study of of the horizontal gauge group to which they are assigned. The motivations
muon-number violating (lepton number conserving) processes. for introducing horizontal gauge symmetries have been many and varied.
It has also been suggested that elementary Higgs fields should be They include attempts to calculate the e-u mass ratio (Barr and Zee, 1978),
discarded and a dynamical mechanism (Weinberg, 1976; Susskind, 1980) fermion mixing angles (Wilczek and Zee, 1979), to get CP violation with
for spontaneous symmetry breaking must be sought to avoid the gauge only four quarks (Maehara and Yanagida, 1979; Davidson et al., 1979a),
t. 
etc. Horizontal gauge bosons also occur in the dynamically broken
theories, and this has motivated their study (Cahn and Harari, 1980}.
Horizontal symmetries have also been studied to avoid the strong CP
problem (Davidson and Wali, 1980). The main motivation for considering
horizontal symmetries seems to be the existence of the generation puzzle
(Harari, 1979; Davidson et al., 1979b). Attempts to explain the repeating
generation structure have also been made by assuming that quarks and
leptons are composite objects and that generations represent excitations.
The problems in constructing a satisfactory theory seem to be far from solved.
Muon number is also violated in grand unified theories (in fact,
both lepton number and baryon number are themselves violated in these
theories). Grand unified theories (GUTs) seek to unify strong, weak
and electromagnetic interactions by finding a simple group which contains
the SU(3) x SU(2)y x U(1) group of strong, weak and electromagnetic in-
teractions as a subgroup (Langacker, 1980; Goldhaber, 1977). A striking
prediction of these theories is proton decay. Early experiments on pro-
ton decay have been done by Gurr et al., 1967 and Reines and Crouch, 1974.
Weinberg (1979) and Wilczek and Zee (1979b) have noted that if the only
new masses (in addition to the masses in the usual SU(3) x SU(2) = u(1)
phenomenology) are grand unification masses (the grand desert scenario)
then baryon and lepton number violation is mediated by local four-fermion
operators that respect the SU{3) x SU(2) x U(1) symmetry. This implies
that (B-L) is conserved to leading order in the grand unification mass
{where B is baryon number and L is lepton number). Lipkin (1980) has
related this conservation of (B-L) to properties of the usual weak iso-
spin assignments of the fermions in the Weinberg~Salam model and to pro-
perties of four-fermion operators. If there is indeed no new physics
between 100 GeV and 1015 GeV then the only interesting processes are

proton decay and (possible) neutrino oscillations. It has also been noted
(Marshak and Mohapatra, 1980b; Kuo and Love, 1980; Marshak et al., 1980)
that if there are intermediate mass scales between present energies and
grand unification energies then even In leading order in the grand unifi-
cation mass (B-L) violating processes like neutron oscillations (n-n) and
no-neutrino double beta decay could occur, Interesting new phenomenology,
including muon number violation, could occur in the Pati-Salam GUTs,
which have low mass unification at a scale of about 100 TeV, They are
reviewed in Pati (1978 and 1979).
In addition to the above motivations for modifications to the GWS
model leading to muon number violation It has been argued by some
physicists that the presence of any absolutely conserved quantity not
related to local gauge invariance is not satisfactory (DeRejula, Georgi
and Glashow, 1975; Mack, 1979). This Is because such conserved quan-
tities (eg. baryon number, lepton number, etc.) predict global symme-
tries in the Lagrangian. These global symmetries require gauge trans-
formations Independent of the coordinates, i.e., they require the com-
parison of directions in internal charge spaces at widely separated
space points simultaneously. DeRejula, Georgi and Glashow (1975) also
argue that a global gauge invariance (and the corresponding conserved
quantity) is not satisfactory for the following reason: If an electri-
cally charged particle falls into a black hole, its memory is preserved
by the electric field outside the black hole. However, the absorption
of a neutrino leaves no trace outside, and it doesn't seem possible to
talk about a conservation law for lepton number in the space outside the
black hole. The difference between the electric charge and lepton num~
ber lies in the fact that the former is a local gauge symmetry, and the
electric field which played a crucial role in the argument is the wave

function of the gauge boson corresponding to this symmetry. The
conservation of lepton number is related to a global gauge invariance
and hence there is no gauge boson corresponding to this symmetry.
In concluding the general discussion we can say that the usual
SU(3) x SU(2) x U(1) phenomenology describes present experiments very
well, and predicts muon number conservation. While there is no over-
whelming reason to expect muon number violation, there are enough moti-
vations to make muon number violation an intriguing possibility to con-
sider. The repeating generation structure of the GWS model and the pro-
liferation of flavours still remains an unanswered mystery and there are
many open questions regarding flavours. The search for violations of
established conserved quantum numbers is interesting for another reason:
If a very weakly coupled force exists in nature, it may not be discovered
for a long time because stronger interactions may mask it. However, if
the weak force violates some conservation law valid for the stronger
interactions, then it would be revealed in processes which violate the
conservation law. This argument is due to Lincoln Wolfenstein. In a
certain sense one may say that the weak interactions were discovered
because they violated the law of immutability of elements.
I will now discuss the phenomenology of different muon number vio-
lating processes. Readers not Interested in details may wish to skim
through these sections and read the conclusions. In most theories with
muon number violation the cause is (a) lepton mixing which arises if
the lepton mass matrix is not diagonal with respect to the gauge eigen-
states {Cheng and Li, 1977; Kobayashi and Maskawa, 1973; Marciano, 1978),
{b) existence of exotic gauge bosons with muon number violating couplings
(Deshpande et al., 1979; Mannheim, 1978; Maehara and Yanagida, 1979) and

(c) scalar Higgs couplings which violate muon number (Bjorken and Weinberg,
1977; McWilliams and Li, 19803). The Feynman diagrams contributing to muon
number violating processes belong typically to one of three classes:
(a) tree diagrams resulting from the exchange of a heavy gauge boson or
Higgs boson (eg. Fig. 2a. This diagram could contribute to K® + ue in
a horizontal gauge model), (b) box diagrams as in Fig. 2b {this diagram
contributes to the process KO + pe in the standard model with a
massive vr), and (c) one loop diagrams like the one in Fig. 2c (this
diagram contributes to the process yu + ey in the same model). These
diagrams are usually suppressed by some heavy masses and reduce to local
interactions to leading order in the heavy masses. Hence muon number
violating processes can be described by local effective phenomenological
currents in a way similar to the phenomenological current X current
description of the usual weak interactions. This is the approach | use
in the discussion. The phenomenological coupling constants described
here will be calculable in different gauge theories. When a photon
mediates muon number violating processes (eg. u + eyyjrtua) + eee) the
effective Hamiltonian describing such processes (eg. processes like
u + eee) will contain non-local terms involving the four-momentum trans-
fer. Such terms can be discussed in terms of the effective coupling
constants defined in the section on yu + ey. Muon number violating pro-
cesses which have been discussed in the literature can be classified as
follows: 1, Leptonic processes: (a) u + ey, (b) u + eee, (c) u + eyy,
(d) We + ep, and (e) neutrino processes. 2. Semi-leptonic processes:
(a) ve” conversion, (b) ue conversion, (c) kaon decays, and (d) no-
neutrino double beta decay. The phenomenology and current experimental
limits on these processes are discussed. below.


\section{\label{sec2}II. Leptonic processes}

The phenomenology of this process was first discussed by Weinberg
and Feinberg (1959). The general form of the matrix element for
the electromagnetic current operator between the states of an electron
and a muon is determined by Lorentz invariance and electromagnetic
current conservation to be of the form (Cheng and Li, 1977a; Weinberg and
Feinberg, 1959; Lee and Shrock, 19773):
Ta = <elpe) UY (0) Julpy)>
Gp — R
= = Te (pe) [fue (1+v5) im,02,a+ EL (14s) (rya-ayv-a) uy (py)
+ right-handed terms, (1)
where q=p,-pe and my is the muon mass. The form factors fy, fg etc.
could be functions of g? but the gq? dependence is generally suppressed
by powers of some heavy masses occurring in the gauge theories. Also,
in many gauge theories only electrons of a particular helicity are in-
volved in this process (see, for example, Cheng and Li, 1977a;
Bjorken and Weinberg, 1977). This means that only left- or right-handed
coupling constants are non-zero in Eq. 1. When the photon is on its
mass shell (i.e., 2=0) only fy and fMR contribute to the amplitude
for uw + ey. The amplitude is of the form:
Ge —
mp + ey) = 7 ve (FHL (1+v5) +f (1-5) )im,o3 aval, (2)
where * is the photon polarisation vector, €tqy=0. The rate for u > ey
2
Gem
rut + ety) = gn (Im 12+) mr 12) : (3)

A complete phenomenological analysis of this decay (including a study of
CP violating correlations) was made by Tung (1977). Recent calculations
are done by Cheng and Li (1980a), Inami and Lin (1980) and Ma and
Pramudita (1980). If this process occurs then a measurement of the
electron asymmetry using polarised muons would give information on the
helicity structure of the lepton number violating current. A search for
this process is quite useful. When electrons of a particular helicity
are produced in the decay the amplitude contains only one term whose
phase is unmeasurable and hence CP violating effects will be absent
(Treiman et al., 1977). Also, in this case the angular distribution of
the outgoing electron with respect to the initial muon polarisation is
of the form [1 t cos(8)] where 8 is the angle between the electron
three-momentum and the muon spin (+ for left-handed electrons and - for
right-handed ones). The experimental upper limit on the rate for this
process is normally quoted relative to the ordinary muon decay rate
Tut > e*vev,) = GEm3/ (1923). The present upper limit on the branching
ratio Rey is 2 x 10710 and this limit has been set at Los Alamos (Bowman
et al., 1979). The experiment consists of looking for an e* and a y-ray
in time coincidence with the three-momenta being collinear and of magni-
tude 52.8 MeV. The main background for this experiment came from ran-
dom coincidences between the et from normal muon decay and a y-ray from
the bremsstrahlung process pt + etvevyy. The latter process also contri-
buted to the background because in some of the events the electron and
the photon were almost collinear and the magnitudes of the three-momenta
were close to 52.8 MeV. There were about a hundred events which satis-
fied the conditions on the momenta within experimental resolutions. All
of them could be attributed to background. A new experiment is under
- 20 ~
way at Los Alamos to improve the limits on the branching ratio for
u+ey, u~+eee and u + eyy, using the crystal box detector.
B. p= + e eet
w= + e"e”et occurs. in all models in which yu + ey occurs, because of
pair production by the photon. In most models there are additional
Feynman diagrams which contribute to this process, for example, box dia-
grams similar to the one in Fig. 2. The pre-gauge theory discussions of
this process were beset by difficulties because the additional diagrams
often involved loop integrations and were calculable only in a renormal-
fzable theory. Bander and Feinberg (1960) calculated the rate for
p= + ete”e” including only the contribution from un + ey (they performed
their calculations In terms of effective u ey couplings. Their effective
couplings are essentially the same as those of Eq. 1). "This process was
also studied by Marciano and Sanda (1977a; 1977b) among others. The
advent of renormalizable gauge theories made it possible to calculate
the contributions other than the photon exchange contribution to this
process. Because of the non-photon exchange contributions the relation
between the branching ratios for u + eee and u + ey becomes model depen-
dent (Marciano, 1978). In general the amplitude for u + eee (excluding
the photon exchange contribution) takes the form (Cheng and Li, 1977a;
Altarelli et aql., 1977):
mikiake) = 2 [Telk)) Oars, wpe] =
[Reta fe + op OBA v6] @
2
where k; and k, are the four-momenta of the electrons, kj is the four-
momentum of the final state positron and p is the four-momentum of the
initial muon. For the sake of simplicity | haven't written down the


terms involving a V+A pe current. In many theories the pe current is
either V-A or V+A and tensor terms are also suppressed by heavy masses
(Cheng and Li, 1977a). The rate for p -> eee has an interference term
between the photonic and non-photonic amplitudes. In terms of the effec-
tive couplings of Eq. } and Eq. 4 the partial width for this process is
2 5
T(u+3e) = ath 1670 (itn pay - 2 If] 2-48ra Re (fife) +
12na| fg |2+|g | 2+2| gg | 2-2/kma Re [(Fg-2710)* (aL 4208) | . (5)
The rate is normally compared to the ordinary muon decay rate. In many
theories one expects g; and gg to be somewhat larger than efg and efy
as the former couplings get contributions from box diagrams which are
enhanced by logarithmic factors (Marciano, 1978). The p~e™ conversion
process which is described later is likely to be a more sensitive probe
of lepton number violation than nu + eee, unless mixing angles conspire
to suppress it. The best upper limit for the branching ratio for
u > eee has been set by Korenchenko et al. (1976).at Dubna. They quote
an upper limit of 1.9 x 1079. The experiment involved stopping positive
pions in a target and looking for two positrons and an electron in time
conicidence using a magnetic spark chamber detector. The process was
searched for using kinematic reconstruction of the event. All the events
which remained after imposing various cuts could be attributed to back-
ground. According to Korenchenko (1976) the possible sources of back-
ground are events from the photonic muon decay followed by internal or
external conversion of the photon, events from accidental coincidences
of positrons emitted by two decaying muons where one of the positrons
is later scattered by an electron of the target material and transfers
an exceptionally large energy to it, and finally events from the
. 
accidental coincidence between emission of a positron by a muon and
passage through the chamber and target of a charged particle from the
region outside the chamber (the passage of the charged particle simu-
lates the production of a positron and an electron).
C. u~+eyy
un + eyy has been studied by Bowman et al. (1978). They point out
that the rate for yu + eyy may be larger than for yu + ey in some models,
in particular for models in which the latter process is suppressed by a
natural mechanism and in which charged exotic heavy leptons exist. They
also point out that experiments setting limits on the branching ratio
for u + ey can be used to set limits on pu +> eyy. In most of the exten=
sions to the standard model that are presently studied the rate for this
process would be of order a times the rate for yu + ey.
D. ey + pe
This process has been studied to test for the multiplicative lepton
number scheme. It also occurs In various extensions of the GWS model.
It has been studied by Feinberg and Weinberg (1961) and by Pontecorvo
(1957). Present experiments can test for the effective coupling con-
stant geff in the phenomenological Lagrangian:
Gp fm am
Leff = gefs 7 (I 0+rs)e) (Iv (4vs)e) 6)
at the level geoff = 1 (Frankel, 1974). The rate for the conversion in
solids and in gases Is quenched below the conversion rate in vacuum,
The muonium-anti-muonium transition does not seem to be a sensitive
test of lepton number violation.
E. Neutrino processes
These tests are difficult because the neutrinos are weakly inter-
acting neutral particles and are difficult to detect and study. In
- 23 ~
addition to the experiment of Danby et al. (1962) and Davis (1955) which
established the existence of two kinds of neutrinos, neutrino experi-
ments have been carried out to test the nature of the neutrinos from
muon decay (Willis et al., 1980) and to look for neutrino oscillations
(Reines, Sobel and Pasierb, 1980; Blietschau et al., 1978). The neu-
trinos from muon decay were studied by Willis et al. (1980) at Los
Alamos. They stopped positive muons and fooked for electron-type anti-
neutrinos from the muon decay. They also looked for electron-type
neutrinos. The former process is allowed by the multiplicative lepton
number conservation scheme while the latter process is allowed by both
schemes. They looked for vg using the reaction vep + net and for vg
using ved + ppe~ (both reactions are allowed by both lepton number
schemes). They find that the branching ratio T(u* + e*vevy)/r (ut + all)
is ~0.001 + 0.04, thus providing strong evidence against the multipli-
cative scheme, They find good agreement with theory for the additive
lepton number scheme.
The possibility of neutrino oscillations (i.e., the nature of the
neutrino changing with time) has been studied for a long time (Maki,
Nakagawa and Sakata, 1962; Nakagawa, Okonogi, Sakata and Toyoda, 1963;
Pontecorvo, 1967; Gribov and Pontecorvo, 1969; Wolfenstein, 1980a; Wolfenstein,
1978; Cheng and Li, 1978; Marciano, 1980; Schechter and Valle, 1980; Bilenky and
Pontecorvo, 1978, and other references in this section). Bilenky and Pontecorvo
(1978) give a good review of neutrino oscillations. The recent litera-
ture on this process is vast and growing, and requires a review of its
own. We will content ourselves here with an explanation of the basic
aspects, which are very simple. Three experimental results motivate
the present surge of interest in neutrino processes. They are the solar
24 -
neutrino experiment (Davis Jr., Evans and Cleveland, 1978), the neutrino
mass experiment {Lubimov et al., 1980), and one of the neutrino oscilla-
tion experiments (Reines, Sobel and Paslierb, 1980). The solar neutrino
experiment has been of interest for many years now. This is an experi-
ment to measure the flux of neutrinos reaching the earth after they are
produced in energy-releasing processes in the sun. The experiment de-
tects a flux which Is a factor of 3 or 4 below the theoretical predic-
tion. The discrepancy seems to require modifying our model of the sun,
or of modifying the standard model of weak interactions. For example,
the discrepancy could be due to neutrino oscillations. The neutrino
mass experiment measures the shape of the electron spectrum in tritium
beta decay 3H + He + e” + v,. The shape Is sensitive to the mass of
the Ve and the experiment indicates that the mass lies between 14 eV
and 45 eV. The neutrino oscillation experiment detects the nature of
the neutrinos coming from a nuclear reactor. About half the neutrinos
seem to change their type in traversing the 11.2 m from the reactor core
(where they are produced) to the detector. These experiments seem to
indicate the need for modifying the standard weak Interaction model,
but they are difficult experiments and are not conclusive.
The presence of neutrino oscillations implies that the neutrinos
have mass and mix among themselves. Neutrino oscillations are very simi-
lar to the K9-K° oscillations, the study of which led to the discovery of
CP violation. They occur when the neutrino mass eigenstates are not
eigenstates of the weak Interaction. In this case the weak interaction
properties of a neutrino are periodic functions of time. The interesting
information about neutrinos is the nature of their mass matrix (i.e.,
masses, mixing angles and Dirac or Majorana nature). A discussion of this

question is given by Cheng and Li (1980b). The mass of a particle can be
either a Dirac mass or a Majorana mass. If the particle has a Dirac mass
the particle and its antiparticle have a separate identity. If the particle
has a Majorana mass then it is its own charge conjugate. [In general the
Lagrangian will contain both Dirac and Majorana masses. The physical par-
ticles are mass eigenstates and one has to diagonalise the mass matrix in
the Lagrangian to see the nature of these particles. For one particle the
general mass term in the Lagrangian takes the form (Cheng and Li, 1980b):
Lass = 0b, bg + AST, + BURup + Hermitean conjugate Nn
where L and R denote left and right hand components of the fields and
the superscript c refers to the charge conjugate field. When the fields
are mixed among themselves to diagonalise the mass matrix and get physical
states, one finds that in general the physical particles are Majorana
particles. One gets the familiar Dirac particles if A=B=0. Thus, the
Dirac mass is a special case of a Majorana mass. When A or B Is not zero
the Lagrangian describes two Majorana particles with different masses
172 {(A+B) + [(A-8)2 + D2]1/2}, If the Lagrangian conserves some charge
(say electric charge or lepton number) and if the particle has a non-
zero eigenvalue for this charge then A and B are forced to be zero. This
explains why Dirac particles are so ubiquitous though they form a partic-
ular case of a more general situation. This also explains why most ex-
tensions of the standard theory of weak interactions give Majorana neu-
trinos if lepton number conservation is not imposed a priori in the
Lagrangian (Cheng and Li, 1980b). Phenomenological implications of non-
zero neutrino masses and mixings have been studied by, among others,
(Ng, 1980 and 1981; Barger et al., 1980a, 1980b; Shrock, 1980;
ve

Kalyniak and Ng, 1981; see also the references in Cheng and Li, 1980b).
Rosen and Kayser (1981) and Frere (1981) mention that neutrino
oscillations affect the interpretation of the sin2e, measurement.
Kalyniak and Ng (1981) discuss the electron and neutrino correlation in
muon decay as a possible probe of the Majorana or Dirac nature of the
neutrinos. Grand unification models provide a motivation for non-zero
neutrino masses and this has been studied in (Gell-Mann, Ramond and
Slansky, 1979; Georgi and Nanopoulos, 1979; Barbieri, Ellis and Gaillard,
1980; Witten, 1980; Wolfenstein, 1980b.). Zee (1980) considers the
possibility of the neutrinos picking up Majorana masses from a charged
SU(2)y singlet in the Weinberg-Salam model. He mentions that such a
Higgs particle could arise in some grand unified theories. Many of the
grand unified models predict neutrino masses in the range 10°5 to 102 ev
(Langacker, 1980).
The experimental bounds on neutrino masses are usually given neglec-
ting the possibility of neutrino mixing. The limits are then 14 eV <
m{ve) < 46 eV (Lubimov et al., 1980), m(vy) < 0.52 MeV (Lu et al., 1980;
Daum et al., 1979) and m{v;) < 250 MeV (Bacino et al., 1979). Simpson
(1981) quotes a value for m(ve) of 20 eV, but with large errors. infor-
mation on neutrino masses can also come from astrophysical arguments.
For example, standard cosmological models predict that the universe is
permeated by neutrinos. |f these neutrinos have non-zero masses, this
would affect the expansion rate of the universe. From the measured ex-
pansion rate one gets the limit that the sum of all stable light neutrino
masses should be less than about 60 eV (Cowsik and McClelland, 1972;
Lee and Weinberg, 1977b; Tremaine and Gunn, 1979). Also, the helium
abundance of the universe constrains the number of distinct light stable

neutrinos to be less than 4 (Shvartsman, 1969; Steigman et al., 1977).
More model-dependent cosmological arguments give more constraints (Dolgov
and Zeldovitch, 1981; Steigman, 1979). | quote as an example an upper
limit on the v; mass of 35 eV calculated by Cowsik (1980). The experi-
mental situation regarding neutrino oscillations is still ambiguous and
many new experiments are being planned. A survey of some experiments is
given by Mann (1981). Improving our experimental knowledge about neu-
trinos is very important for indicating the direction in which particle
physics theory should develop. Neutrino experiments also have important
implications for astrophysics and cosmology.

\section{\label{sec3}I1I. Semi-leptonic processes}


A. u"e” conversion
When negative muons are stopped in some material they are captured
by atoms and form muonic atoms. {in a time short compared to their 1ife-
time they cascade down to the ground state atomic orbit. From this state
they are either captured by the nucleus with the emission of a neutrino
or they decay into an electron and two neutrinos. !f muon number is not
exactly conserved the ye conversion process u™+(A,Z) + e“+(A,Z) can occur
some of the time. The phenomenological study of this process has a
long history beginning with the work of Weinberg and Feinberg (1959).
This process has some similarities to yu + eee, For example, the discus-
sion of Marciano and Sanda (1978, 1977a, 1977b) on the relation between
u~> ey and yu > eee is also relevant for this process. With a suitable
choice of stopping material the rate for ye conversion becomes larger
than the rate for p > ey or py + eee in most theories. This is because
the process can take place coherently (i.e., the nucleus is not excited
and remains in the ground state). The overlap integral of the muon

wave function with the nuclear density occurring in the matrix element
further enhances the rate. Finally, just as in the case of pu -> eee the
nonphotonic contributions to the rate are enhanced over the photonic
contribution because of logarithmic factors coming from box Feynman
diagrams (Marciano, 1978). Thus, ue conversion seems theoretically to
be a good place to look for muon number violation. This has also been
emphasized by Altarelll et al. (1977). An interesting aspect of the pe
conversion process is that a careful measurement of the branching ratio
for different elements reveals the nature of the current mediating the
process, and hence throws light on the flavour violating mechanism.
For example, flavour violation due to gauge bosons, Higgs bosons or
lepto-quark pseudoscalars (occurring in dynamically broken theories)
leads to vector currents, scalar currents or a combination respectively,
In all three cases the calculations of a recent phenomenological study
(Shanker, 1979) can be used to get the coupling constants of the
theory from the pe conversion branching ratios for a few elements.
Experimentally also the pe conversion process has many advantages
as a probe for muon number violation. The experimental signal for the
coherent conversion process is very clear because it involves the detec-
tion of a single monoenergetic electron of energy Ey (where Ey is the
muon energy, i.e., muon mass minus the muon binding energy). The coher-
ent process dominates the rate in general, and since it involves the
detection of a single particle in contrast to p + ey or u > eee, the
experimental search is facilitated. The background to the coherent
process (from electrons coming from bound muon decay and from conversion
of photons coming from radiative pion and muon capture) is negligible.

All these features make the process an attractive probe for muon number
violation,
The ue conversion rate is normally compared to the ordinary muon
capture rate. The present limits on the branching ratios are 7 x 10711
for sulphur (Badertscher et al. quoted in Egger, 1979; see also
Badertscher, 1977) and 1.6 x 1078 for copper (Bryman et al., 1972). An
experiment to measure the branching ratios for different elements with
greater accuracy is under way at TRIUMF in Vancouver. The TRIUMF group
use a Time Projection Chamber to increase the solid angle acceptance and
to improve the energy resolution over previous experiments.
B. wu"e* conversion
in y"et conversion the muon in a muonic atom gets converted into a
positron while two of the protons in the nucleus become neutrons. It
thus involves a double charge exchange between leptons and hadrons. It
is allowed in the Konopinski-Mahmoud scheme of lepton numbers. In this
scheme it is suppressed due to the helicity assignments of the neutrino
and the V-A nature of the weak interaction (Frankel, 1974; Pontecorvo,
1967). The process could also occur due to Majorana neutrinos (Kamal
and Ng, 1979) or Higgs particles (Vergados, 1981). It was first studied
in detail by Kamal and Ng (1979). The calculations of the nuclear matrix
elements needed for the process are similar to those done for the no-
neutrino double beta decay, which is reviewed in a later section. The
similarity of the calculations stems from the fact that both processes
involve double charge exchange. The experimental limit on the rate
for u"e* conversion is normally quoted as a fraction of the ordinary
muon capture rate. Experiments which set limits on u”e” conversion
also set limits on the branching ratio for this process. Bryman et al.

(1972) report a limit R(u~e*) < 2.6 x 108 for copper. A limit R{u~e*) <
9 x 10"10 has been set by the Bern group (Badertscher et al., 1978) for
sulphur. For experiments which detect the final state positron the back-
ground comes from radiative muon capture followed by photon pair produc-
tion. An interesting new way to observe this reaction is to look for
the recoiling nucleus. Such an experiment was done by the Basel-
Karlsruhe group at SIN (Abela et al., 1980). They searched for 127Sb
from u~et conversion in 1271 using radiochemical techniques. They get
the limit R(u"e*) < 3 x 10710 for iodine. The background for this ex-
periment is production of !27Sb through picnic double-charge exchange.
For the pion contamination of <5 x 10™% in the u~ beam this was found to
be negligible.
C. Muon number violating kaon decays
The decays normally investigated are the decay of $K_L$ or Kg into ue
or ye with or without a pion and the decays of charged kaons into pions
and an electron and a muon. Experimental limits are available only for
KL + ve or pe, K* + eurt and K* + rje. Muon number violating decays
can be discussed in terms of the effective Lagrangian:
A 1 =p! ' i
Letr = =[er (F1(5)*F1 (5) Ys )udl gy eT (Fi a*f @ sya | +
+ Hermitian conjugate . (8)
where
Its) =5rld + rls,
Sa = 5rd - Tris, (9)
and Ti are I, Ys, vA, ys and gWV for i = S,P,V,A,T respectively.
Herczeg (1979) pointed out in a phenomenological study that muon number
violating kaon decays are likely to be small unless the flavour violating

mechanism is suppressed for the Ko-Ko transition. Models are known in
which the latter process is suppressed and in which muon number viola-
ting kaon decay rates can be interesting (Maehara and Yanagida, 1979;
Cahn and Harari, 1980; Shanker, 1980 and 1981).
Let us consider the experimental constraints on the phenomenological
coupling constants in Eq. 8 from the process Ki + u*e™. Only the axial
vector and pseudoscalar ''symmetric' currents with coupling constants
fa(s)xxx FA(s)s fp(s) and f(s) will contribute to the process. If we
define the Ki to vacuum matrix elements
<0] Rs) IKL(P)> = Pyman/ 2m
<0 ss) KL (P)> = -mBap//Zmy, (10)
the rate for $K_L$ + u*exxxxx can be written (Herczeg, 1979)
rk + pte”) = mg {1-mZ/mg) 2 (|A]2+|8)2) / (8n) an
where
Am (h.2 x 1077) fp (sya = (2 x 1078) fp (gap,
Bs (4.2 x 1077)F5 5yap + (2 x 1078) Fp (5) ap. (12)
The branching ratio for K+ p*e” is
$B(K_L + wexxxx)$ = T(K, + ptexxxx )/T(K + all)
= (1.4 x 1015) (jA]2+|8B]2). (13)
The experimental upper limit for the branching ratio (Clark et al., 1971)
is 2 x 1072 and implies that the absolute value of A (or B) should be
less than 1.2 x 10712, The form factor ap can be related to the leptonic
kaon decay form factor fy using isospin symmetry:
<0 4} (oy [KL (P)> = vZ<O[SyyysulK*> = VT FyPy/ VER xxxxx (14)
which gives ap = v2 fy/mk. ap can be estimated from ay using the

current algebra relation (Branco et al., 1976)
As) = 1 (mg+mg) ofs) (15)
which gives
ap = ap mg/ (mg+mg) (16)
The above relations give {(Herczeg, 1979)
ap = 0.48
ap = 1.5 (17)
If we take fi = 1.23 my and ms = 150 MeV, mg = 7.5 MeV (Weinberg, 1977b).
The decay could occur due to neutrino mixing, for example it could occur
in the standard model with six leptons and six quarks if the t neu-
trino were not massless and mixed with the other neutrinos. In the
standard model all muon number violating effects depend on the parameter
By where B and y measure the amount of v; mass eigenstate in the gauge
group eigenstates vg and v) respectively (Lee and Shrock, 1977a; Cheng
and Li, 1977b). The KL + ue rate in this model has been calculated by
Lee and Shrock (1977a) and by Cheng and Li (1977b). Lee and Shrock find
that the main contribution to the rate comes from the free quark diagram
Fig. 2b and the contribution from the intermediate two photon decay
KL + 2y + pe is small, This is in contrast to the situation for the decay
KL + un where the dominant contribution comes from the 2y intermediate
state. The ratio of the free quark contributions to $K_L $ + pe and to
KL + up is of the order of (8y)2. Hence the ratio of the rate for
KL + ve to the total rate for $K_L $ + uy is of the order of 1073(gy)2.
Since the branching ratio for $K_L $ + up is very small it is clear that the
rate for K, + ue is negligible in the standard model. With the experi-
mental Vimits (By)? < 2 x 103 (Lee and Shrock, 1977a; Altarelli et al.,
- 33 ~-
1977; Cheng and Li, 1977b; Bleitschau et al., 1978) and my, < 250 MeV one
obtains
I (Kg > utexxxxx)
r(Kk xxxxx all)
<5 x 10716, (18)
which is far below the present experimental limit. More interesting
limits are obtained in horizontal gauge models and technicolour models.
The only other muon number violating kaon decays for which signifi~
cant limits exist are K + re and K* + nye. These processes could
occur due to "'symmetric' and "antisymmetric" scalar, vector and tensor
currents in the effective Lagrangian Eq. 8. Using isospin symmetry the
K* to nt vector matrix element can be related to the K' to 70 matrix
element which is known for vector currents from K* + moutyy and K+ +
mOetye decays. For scalar currents the matrix element can be related to
the vector matrix element using current algebra (Branco et al., 1976).
The present limits on the branching ratios are (Diamant-Berger et al.,
1976) :
+ + tT
r{k* +xxxxx ety*rt) <7 x 109,
r{k* ~ all)
(Kt +xxxx gtyute)
T{K+ = all)
<5 x 1079, (19)
For illustrative purposes | will describe the phenomenological cal-
culation of Herczeg (1979) on the rate for Kt + n*ye mediated by a scalar
(Higgs) particle. He assumes that a scalar particle  couples to fermion
currents with coupling constants (g'ue + g"4%5) 0), where Hs) is defined
by Eq. 9. With these couplings the effective muon number violating
Lagrangian due to the exchange of the scalar particle is given by Eq. 8
with Gpfg(s)/vZ = g'g"/m§. The branching ratio for K* + nye is then
given by 8|fg (sy |2- One can get some constraints on fs (s) because the
scalar particle exchange also contributes to the Ki -Kg mass difference.
. 
The effective AS = 2 Lagrangian due to the scalar particle exchange is
of the form (g"/my) 2 (sd) (sd) and this contributes to the K -Kg mass
difference an amount
amg = 2(g"/my)? <K°|Sdsd|Kko> . (20)
Herczeg had to estimate the K°-KC transition matrix element in Eq. 20.
A study of the Ko-K° transition matrix elements for a general combination
of the AS = 2 currents now exists (McWilliams and Shanker, 1980b), and
| will use the results for the analysis.
| will describe the calculation of the KO-K° transition matrix ele-
ments briefly in view of their importance. Three methods have been
used to calculate them. One method is to use the bag model (Shrock and
Trieman, 1979). The remaining two methods are both called the vacuum
insertion method. To avoid confusion | will use different names. The
first evaluation of the K9-K° matrix element was done for the (V-A)
AS = 2 current which occurs in the standard weak interaction model
(Gaillard and Lee, 1974). They used what is called the vacuum inser-
tion method, and | will call this method the Fierz transform method.
In this method the kaon is treated as a two quark object (i.e., the
effect of sea quarks and gluons are ignored). The four quark AS = 2
operator is broken up into creation and annhilation components, and these
are Fierz transformed such that the operators destroy the quark-anti-
quark pair in the KO meson before creating those in the KO meson. If
a complete set of states is now introduced between the operators, only
the vacuum state will contribute provided the K° meson is really made
up of only a quark and an anti-quark. That is why this method is called
the vacuum insertion method. This method relates the KO-K° matrix

element to the K° to vacuum matrix element which is experimentally deter-
mined. The third approach is also called the vacuum insertion method
because it is often confused with the method which | described above,
the Fierz transform method. In this method a complete set of states is
introduced without making the Fierz transformations. Since only the
kaon to vacuum matrix element is known to reasonable accuracy, the sum
over complete states is truncated with the vacuum contribution. Hence
the third approach can be called a truncated complete states method. In
the study of McWilliams and Shanker the Fierz transform method of calcu-
lation and the bag model calculation are used for a general combination
of AS = 2 currents, unlike the previous studies which were limited to
a (V-A) combination of currents. This study finds that the bag model
and Fierz transform methods are in reasonable agreement.! However, the
truncated complete states method, which is commonly used, can give very
misleading results due to the neglect of states other than the vacuum
in the sum, - )
I will use the Fierz transform value for the matrix element
in Eg. 20. The Fierz transform calculation is sensitive to current
algebra quark masses, and | use the values mg = 150 MeV, md =
7.5 MeV (Weinberg, 1977b). The matrix element takes the value
-0.01 GeV3. Assuming that the scalar particle contribution to the
1The bag model matrix elements are approximately 0.7 times the Fierz
transform matrix elements, and this is within the expected accuracy of
the methods. Note that for the (V-A) case the bag model calculation is
0.4 times the Fierz transform calculation. The agreement is poorer for
the (V-A) case because a cancellation occurs between the vector and
axial vector matrix elements, and hence the results are more sensitive
to the bag model parameters.

KL ~Ks mass difference is comparable to the experimental value of 3.5 x
10715 GeV, we find that my > g"(2 x 106) GeV. Often the coupling of
scalar (Higgs) particles to fermions is proportional to the fermion
masses. Hence, if we assume that g'/g" = my/mg we find that the above
result constrains the K* + ntye branching ratio to be less than 1.5 x
10715. For g'/g" = my/mg the branching ratio is less than § x 10~1!3,
These are well below the experimental value quoted in Eq. 19, illustra-
ting the strong constraints provided by the Ki ~Kg mass difference on
lepton number violation in some models. The branching ratios calculated
above are two orders of magnitude larger than the results of Herczeg.
This is because | used the result of a KO-K° matrix element calculation
instead of an estimate. The large change reflects the fact that the
Ki -Ks mass difference is proportional to the amplitude of the flavour-
violating current and hence is inversely proportional to ms while the
rate for K* + ntue is proportional to the square of the amplitude and
hence Is inversely proportional to mg. Thus, an order of magnitude
change in the value of the K°-K° transition matrix element that was
used changed the estimate for the K* + nue rate by two orders of mag-
nitude. The use of better K°-K® transition matrix elements does not
change the results of the other cases considered by Herczeg so
drastically.
The search for muon number violating kaon decays is useful as a
probe for horizontal gauge bosons or the leptoquark pseudoscalars of
dynamically broken theories. Doubly charged muon number violating kaon
currents have also been discussed in the context of the Konopinski-
Mahmoud lepton number scheme. However, p~et conversion and no-neutrino
double beta decays seem to provide more stringent tests of doubly charged
muon number violating currents (Frankel, 1974; Diamant-Berger et al., 1976).

Neutrinoless double beta decay
The neutrinoless double beta decay process is the emission of two
electrons by a nucleus, with no accompanying neutrinos. Two neutrons in
the nucleus get converted into protons. This process violates lepton
number. It could occur in gauge theories if the neutrinos pick up
Majorana masses or it could occur due to Higgs particle Interactions.
A recent review of the process is given in Rosen (1980). if the neutrino
is a Majorana particle and has zero mass the process is not allowed
because of the (V-A) nature of the weak interaction. This is because
the process occurs in two steps, i.e., n + pte +yy followed by vy+n +
pte”, and the neutrino emitted in the first step has the wrong helicity
to participate in the second step. The process could occur |f the weak
interaction had right-handed components, or if the neutrino had non-
zero mass, since mass terms lead to violation of helicity. It is some-
times stated that a Dirac particle and Majorana particle are equivalent
in the zero mass limit. This is only true for the equation of motion,
i.e., the Dirac and Majorana equations become equivalent for zero mass.
As the above discussion shows, even In the zero mass limit the Dirac or
Majorana nature of the neutrinos could affect thelr interactions.
The weak interaction Hamiltonian for double beta decay with Major-
ana neutrinos has been discussed by Pauli (1957), Pursey (1957), Liders
(1958), Enz (1957). The effect of neutrino mass In overcoming the
suppression due to the helicity of the weak current has been discussed
by Enz (1957), Greuling and Whitten (1960) and Halprin et al. (1976).
Calculations of the lifetimes for the process in terms of weak inter-
action parameters were performed by Furry (1939), Touschek (1949),
Primakoff (1952), Konopinski (1955) and Primakoff and Rosen (1959, 1965

and 1969). There are very large uncertainties (about two orders of mag-
nitude) in the calculations due to uncertainties in the knowledge of
the nuclear matrix elements. Recent calculations of the matrix elements
for the tellurium nucleus were performed by Haxton et al. (1979). All
the above work assumed that no-neutrino double beta decay occurred due
to a Majorana neutrino. The process must involve two nucleons because
one nucleon cannot change its charge by two units (a nucleon has isospin
1/2 and hence only two isospin components). Experimental limits con-
strain the neutrino mass to be either very small or very large. In the
former case the helicity of the weak interaction operates in suppressing
the rate for the process and in the latter case the suppression is due to
the Yukawa-type force generated by the exchange of the neutrino between
two nucleons. If the nucleus contains nucleon resonances in addition
to unexcited nucleons, the process could involve only one resonance of
isospin greater than 1/2. This is because such a resonance would have
more than two components and could change its charge by two units. The
process is assumed to involve two quarks within the resonance (Primakoff
and Rosen, 1969; Halprin et al., 1976). The lower limit on the neutrino
Majorana mass for the heavy neutrino case becomes much more stringent
if such a nucleon resonance is contained in the nucleus. This is be-
cause the quarks participating in the process come closer together than
in the case involving two nucleons. Also, unlike nucleons the quarks
are expected to have no hard-core potential between them. Pontecorvo
(1968) discussed a AL = 2 superweak interaction mediating the process,
in analogy with the superweak CP violating interaction proposed by
Wolfenstein (1964).

The experimental search for the process involves both geo-chemical
and direct detection methods. The lepton number conserving two neutrino
double beta decay process does occur and has to be separated experimen-
tally or subtracted theoretically. A review of the experiments is given
in Bryman and Picciotto (1978). They find that phase space calculations
do not account for the ratio of Tel30 and Tel2?® lifetimes, assuming
that the nuclear matrix elements for Tel3C + Xel30 and Tel30  -> xxxxx xel28
are equal and that the decays are due to lepton number conserving two
neutrino double beta decay modes. They find that the existence of a
small number of no-neutrino decay mode events could remove the discre-
pancy. The uncertainties in the nuclear matrix elements are too large
to draw any definite conclusions regarding lepton number violation
(Haxton et al., 1979).


\section{\label{sec4}Model predictions}

Having seen the types of models which predict muon number violation
and the different muon number violating processes that have been studied,
it is natural to ask which of the processes are most sensitive to vio-
lation of muon number. Unfortunately, the answer to this question is
model-dependent. At present we cannot convincingly say whether muon
number violation should occur at all, the scale at which possible vio-
lations may be expected, or the process most sensitive to muon number
violation. However, one can look at the different types of models and
draw some conclusions,
A. Horizontal gauge models
Let us first look at horizontal gauge models. In these models the
muon number violating rates will be uninterestingly small unless the
contribution of the horizontal gauge bosons to the K -Kg mass differ
ence is suppressed. This is because the K;-Kg mass difference is

sensitive to the matrix element of the flavour violating current, unlike
the rates which are sensitive to the square of the matrix element. A
class of models is known (Maehara and Yanagida, 1979; Cahn and Harari,
1980; Shanker, 1980 and 1981) In which the Kj -Kg mass difference is
suppressed and in which the muon number violating rates could be close
to the experimental upper limits. The suppression mechanism depends on
the fact that the horizontal gauge bosons are almost degenerate in mass.
Table | shows a representative calculation of the constraints imposed
on the parameters of this class of theories by muon number violating
processes. The cases referred to in the table are described in Shanker
(1981) and correspond to different choices of a discrete symmetry in
the model. The muon number violating processes constrain two parameters
in the theory, which are essentially the horizontal gauge boson mass M
and the gauge boson degeneracy breaking, 6 = AM/M. § is expected to be
of order MG/m2 If all gauge coupling constants are comparable (My Is the
mass of the ordinary W boson mediating weak interaction processes). In
reading the table it should be kept in mind that all mixing angles are
assumed to be of order one and possible suppressions due to mixing
angles can change some of the conclusions. For comparison with the table
we note that the imaginary part of the Ki -Ks mass difference constrains
6 to be less than 4 x 10713(M/My)2. Since § Is expected to be of order
M3IMZ, we see that p~e” conversion provides the most stringent constraint
on the theory, i.e., M > 3000 My, unless mixing angles suppress the
rate for this process. The pattern in the table can be understood quite
simply: the mechanism which suppresses the K -Kg mass difference also
suppresses muon number violating rates involving only two flavours, i.e.
u + ey and p + eee. However, the mechanism works only when certain

restrictions are satisfied by the fermion mixing angles, and hence
pu + ey and p + eee are not suppressed in some cases. In these cases
they give strong constraints on the horizontal gauge boson mass M. In
all cases pe conversion provides one of the best limits on M. This is
a general property of ye conversion and is not limited to the class of
horizontal gauge models we are considering. The enhancement of ue con-
version rates in many models reflects the fact that the quarks in the
nucleus contribute coherently to the pe conversion amplitude. It is
possible that the coupling constants conspire to reduce the rate for pe
conversion.? This means that all the processes in the table must be
investigated to look for muon number violation.
Hypercolour models
| now consider hypercolour models, In which the symmetry is dynam-
ically broken. In these models muon number violation can occur due to
light scalar particles (called pseudo-Goldstone bosons. They are actu-
ally bound states of hyperfermions. However, they behave like spin zero
particles at low energies.) Muon number violation could also occur due
to horizontal guage bosons which have to be introduced to give masses
to the ordinary fermions. In hypercolour models the masses of the par-
ticles mediating flavour changing processes cannot be made arbitrarily
large. They are constrained by the {ordinary) fermion masses and mixing
angles. This causes some difficulty for hypercolour models, and they
tend to predict too large a value for the Ki-Kg mass difference. A
2In general this is unlikely, but one can adjust parameters to build a
model where this happens. Since the process involves both up and down
quarks, It Is difficult to adjust parameters such that the process is
completely suppressed.
. 
suppression mechanism for this process seems to be required. In the
case of the flavour changing scalar particles Ellis et al. (1980)
achieve this by what they call "monophagy', namely, each charge sector
of ordinary fermions gets its mass from a unique effective scalar (hyper-
fermion condensate). This is similar to the standard model of weak
interactions with two or more ordinary Higgs doublets and natural fla-
vour conservation. Ellis et al. note that even in monophagous models
coloured leptoquark scalars exist which mediate the Ki > ue process.
Because of fermion mixing one would expect other processes like pe con-
version to also occur, and this Is now being studied (Ng and Shanker,
work in progress). The leptoquark nature of the intermediate boson
implies that the flavour violating current has both scatar and vector
components, and this is a distinctive signal. The nature of the current
would reveal itself in the variation of the ue conversion rate for differ-
ent elements, for example.
For the horizontal gauge bosons in hypercolour models no suppression
mechanism is known for the K -Kg mass difference, but such a mechanism
seems to be needed. One would then expect the phenomenology of flavour
violation due to the gauge bosons to resemble the phenomenology of the
horizontal gauge models considered above. Dimopoulos and Ellis (1980)
have made a useful study of flavour violation in hypercolour models.
They arrive at different conclusions. They argue that in the absence
of fermion mixing the mass of the horizontal gauge boson which gives mass
to a particular quark is inversely proportional to the square root of
the quark mass. Because of the large hierarchies in quark masses this
leads to gauge bosons with very different masses. They then assume that
flavour violating processes are mediated by gauge bosons with masses of

the same order as those which give masses to the quarks participating
in the process. Because fermions do mix, it may be too restrictive to
use flavour changing gauge bosons with very different masses for differ-
ent flavour changing processes, as is done by Dimopoulos and Ellis. The
lighter horizontal gauge bosons, which give masses to. the heavier quarks
(say to the bottom quark), may contribute to flavour changing processes
involving light quarks only (say to ue conversion with a current coupling
to the down quark), because of fermion mixing. in fact, a mechanism
which generates both quark masses and quark mixing angles may not even
have a large hierarchy of horizontal gauge boson masses. The work of
Dimopoulos and Ellis seems to indicate that $K_L $ + pe Is the most sensitive
muon number violating process in hypercolour theories. However, as !
have just argued, a more detailed understanding of the mechanism gener-
ating fermion masses and mixing angles seems to be required before re-
liable conclusions can be drawn. Normally quark mixing angles do not
change the phenomenology by many orders of magnitude. In the work of
Dimopoulos and Ellis, however, the hierarchy of gauge boson masses makes
the phenomenology sensitive to the mixing angles. The mixing angles
which occur in the phenomenology are not necessarily the ones which
occur in the ordinary weak interaction (the Kobayashi-Maskawa or gener-
alised Cabibbo angles).
C. Grand unified theories
An interesting question is the implication of grand unified theories
(GUTs) for muon number violation. The superfluous proliferation of fam-
ilies is as much of a mystery in these theories as it is in the low energy
standard model. {In GUTs also horizontal gauge symmetries have been con-
sidered to deal with the generation problem, and these attempts are

reviewed in Langacker (1980). These attempts include the use of large
simple groups or products of simple groups with discrete symmetries im-
posed to get one coupling constant. The Implications of these models
for the processes considered here depend on whether the horizontal gauge
bosons get masses of the order of GUT masses, or whether they get masses
low enough to give measurable rates for muon number violating processes.
No satisfactory model which accounts for the repetition of generations
exists. Many GUTs do predict non-zero mass for the neutrinos, and hence
have implications for neutrino experiments. The discovery of muon num-
ber violation in other processes would have implications for GUTs, since
the models would have to account for the new phenomena. A class of GUTs,
the Pati-Salam models, have light (10*-106 GeV) leptogquarks which could
mediate muon number violating semileptonic processes like K+ ue and
ue conversion. While no detailed studies of this aspect have been made,
the quark coherence effect in ue conversion should make it a favoured
process. Because of possible mixing angle suppression in pe conversion
muon number violating kaon decays could also be interesting processes.
D. Neutrino mass
Non-zero neutrino masses also imply the existence of flavour viola-
ting processes other than neutrino oscillations. However, the rates will be far
below present experimental limits (Petkov, 1977). Cheng and Li (1980b)
have studied the relation between y + ey and neutrino mass terms in
models with three generations of fermions, which reduce to the standard
model at low energies. They find that in some cases these models can
have detectable yu + ey rates if the model parameters are fine-tuned.
The presence of new mass scales for the neutrinos (in the left-right
symmetric models, or in a model with more than three generations, for

example) could lead to detectable flavour violating rates. Even in the
left-right symmetric models, which have been thoroughly studied, no com-
plete calculation of muon number violating (lepton-number conserving)
rates have been made.
E. Higgs models
Let us finally look at extensions of the standard low energy model
which introduce new scalar (Higgs) particles. The new particles have
been most often introduced to get spontaneous CP violation, or are moti-
vated because GUTs may reduce at low energies to theories which have
these new particles, In theories which introduce new Higgs su(2)y
doublets one can impose a discrete symmetry which enforces natural flavour
conservation. Even when natural flavour conservation is not Imposed
the stringent limit on the K -Kg mass difference is likely to constrain
the Higgs particle masses so strongly that muon number violation is very
small. However, if the contribution of the Higgs particles to the Ki -Kg
mass difference is suppressed for some reason the muon number violating
rates could be close to present experimental limits. For example, in
the model of Lahanas and Vayonakis (1979) the Higgs particle contribution
to the K -Kg mass difference is proportional to the Higgs particle mass
differences. Hence muon number violating processes could have detectable
rates if the Higgs scalars have almost degenerate masses. Bjorken and
Weinberg (1977) have pointed out that in a model with Higgs-particle-
mediated flavour violation two-loop diagrams may make a larger contri-
bution to u + ey than one-loop diagrams. In the model of Lahanas and
Vayonakis p + ey could be suppressed by Higgs particle mass differences
since it involves only two kinds of flavours, like the KO-KO transition.
Processes like ye conversion, on the other hand, may have detectable
4 
rates. The model of Lahanas and Vayonakis makes some assumptions regar-
ding mixing angles (eg. that mixing angles in the left and right-hand
sectors are equal, up to a diagonal phase matrix). The role of these assump-
tions in the suppression of the Ki-Ks mass difference has to be studied..
After the detailed discussion of flavour violation In different models
that has been presented we are in a position to make a general summary.

\section{\label{sec5}Conclusions}

Vv. Conclusions
Many extensions of the standard model that are presently being
studied have non-zero rates for muon number violating processes. A com-
plete study of muon number violation has been made for only some of the
models, On general grounds one finds that neutrino oscillations and pe
conversion are sensitive probes of muon number violation. Neutrino oscil-
lations are sensitive because the experiments measure the flavour viola-
ting amplitudes directly, while the experiments for the other processes
measure the square of the amplitude. pe conversion is interesting be-
cause muon number violation could occur while the neutrinos remain mass-
less. It Is a sensitive process because the quarks in the nucleus con-
tribute coherently to the amplitude. Since mixing angles may suppress
ue conversion one should also investigate the other processes listed in
Table I. For models where both muon number and total lepton number are
violated (models with Majorana neutrinos, which have been investigated
extensively, belong to this class) neutrino-anti-neutrino oscillations
seem to be very sensitive tests. The pe’ conversion and neutrinoless
double beta decay processes should also be investigated because they may
be sensitive to regions of the model parameters which have not been
probed by neutrino oscillation experiments.

\section*{Acknowledgments}
The author thanks Lincoln Wolfenstein for getting him interested in
the topic, and for his ungrudging help and his willingness to share with
the author his vast knowledge of and insight into weak interactions.
Many very useful discussions with him went a long way in making this
work possible, His encouragement during the course of the work is grate-
fully acknowledged. The author also thanks Ling-Fong Li and John Ng
for several helpful discussions, Ling-Fong Li's clear and simple explan-
ations on many difficult aspects of gauge theories were very useful in
addition to being a pleasure to listen to. John Ng's comments on neutrino
processes, p~et conversion and neutrino-less double beta decay were very
helpful. The author thanks Lorraine Gray for her excellent typing.
A major portion of the work was done at Carnegie-Mellon University
for the author's doctoral thesis. The author thanks Lincoln Wolfenstein
and the High Energy Theory group of the Physics department at Carnegie-
Mellon for support during the course of the work. The work was finished
at TRIUMF and the author thanks the theory group at TRIUMF for support
while the work was being completed. .


Table I. Typical constraints on horizontal gauge models from muon num-
ber violating processes. M and My are the horizontal gauge boson mass
and weak gauge boson mass respectively. 6 is the horizontal gauge
boson mass degeneracy breaking parameter, AM/M. 6 is expected to be of
order MG/M2. See text for further details.
Limits on M
Cases |, 11, v, vi Experimental
Lases 1, 1, v, vi
Process Parameter Limit for cases Hi upper limit
5 <4 x 1077 (M/Ny)2 1500 My 2 x 10-10
§  <1076 (H/My)2 1000 My 2 x 10-9
M >3000 My ha 3000 My 7 x to-11
. (sulphur)
r($K_L$ + eu or je)
-9
CET M >100 My 100 My 2 x10
M2300 My 300 My 5 x 10-9
ub >300 Hy 300 My 7 x 107xxx
8These processes require axial vector quark currents. For the cases
listed in the table such axial vector currents arise only at the one-
loop level! and hence the processes are suppressed by factors of coupling
constants. When axial vector currents arise due to tree diagrams the
limit on M is much more stringent, M > 1000 My. For example, this
would be the limit for the case in Shanker (1981) where the left and
right mixing angles for the d type quarks differ by complex phase
factors.
bEor cases (i) and (v) some mixing angle relations result in a suppres-
sion of this process. The relevant bound In these cases is on §,
6 < 1075(M/m,) 2,

Addenda
In the section on pe conversion I should mention that Lincoln
Wolfenstein [1977, in Proo. of 7th Int. Conf. on High Energy Phyeica and
Nuclear Structure, edited by M.P. Locher (Birkhduser Verlag, Basel), P—
p. 363] emphasized the reliability with which information on gauge theory
parameters can be deduced from experimental limits on the ue conversion
rate, He played an important role in the nuclear physics calculations
(a)
which are required for u€ conversion. In the section on neutrino pro-
cesses | should mention that Douglas R.0. Morrison (1980, CERN Report
CERN/EP 80-190) gives a critical discussion of the experimental evidence
for neutrino oscillations and mentions that explanations not Involving
neutrino oscillations seem to exist for all the experimental results.
He also discusses anomalous results in the CERN beam dump experiments
and mentions that explanations involving neutrino oscillations are un-
likely for these results.
Epilogue
We began the discussion with the five elements of the ancients, pro-
gressed to ninety-two and beyond, went back to two, we are progressing
again, and the end is not in sight, The riddle we have discussed Is old,
older still than the sphinx's riddle: what creature, with only one voice,
has four legs in the morning, two at midday, and three In the evening,
and yet is weakest when it has the most? Many weary travellers have
attempted to answer the former riddle, and yet we walt for the Oedipus
who will make the sphinx cast herself from her rock and cease to
torment us!
Representative Feynman diagrams contri-
Fig. 2.
A triangle anomaly,
Fig. 1.
buting to muon number violation in different models.




\bibliographystyle{amsplain}
\begin{thebibliography}{10}

\bibitem{Abela}Abela, R. et al., 1980, Phys. Lett. 958, 318.

\bibitem{Abers}
Abers, E.S. and B.W. Lee, 1973, Phys. Rep. 3, 1.


\bibitem{}
Adler, S.L., 1970, Lectures on Elementary Particles and Quantum Field
Theory, edited by S. Deser, M. Grisaru and H. Pendleton (MIT Press,
Cambridge).

\bibitem{}
Altarelli, G. et al., 1977, Nucl. Phys, B125, 285,

\bibitem{}
Anderson, C.D. and S. Neddermeyer, 1938, Phys. Rev. 54, 88.

\bibitem{}
Bacino, W. et al., 1979, Phys. Rev. Lett, 42, 749.

\bibitem{}
Badertscher, A. et al., 1977, Phys. Rev, Lett. 39, 1385.

\bibitem{}
Badertscher, A. et al., 1978, Phys. Lett. 798, 371.

\bibitem{}
Bander, M. and G. Feinberg, 1960, Phys. Rev. 119, 1427.

\bibitem{}
Barbierl, R., J. Ellis and M.K. Gaillard, 1980, Phys. Lett. 908, 249,

\bibitem{}
Bardeen, W.A., 1974, Nucl. Phys. B75, 246.

\bibitem{}
Barger, V. et al., 1980a, Phys. Rev. Lett. 45, 2084.

\bibitem{}
Barger, V. et al., 1980b, Phys. Rev. D22, 1636.

\bibitem{}
Barr, S. and A. Zee, 1978, Phys. Rev. D17, 1854.

\bibitem{}
Bilenky, S.M. and B. Pontecorvo, 1978, Phys. Rep. Mic, 225.

\bibitem{}
Bjorken, J.D. and S. Weinberg, 1977, Phys. Rev. Lett. 38, 622.

\bibitem{}
Blietschau, J. et al., 1978, Nucl. Phys. BI33, 205.

\bibitem{}
Bludman, S., 1958, Nuovo Cimento 9, 433.

\bibitem{}
Bowman, J.D., T.P. Cheng, L.F. Li and H.S. Matis, 1978, Phys. Rev. Lett. 41, baz.

\bibitem{}
Bowman, J.D. et al., 1979, Phys. Rev. Lett. L2, 566.

\bibitem{}
Branco, G.C., T. Hagiwara and R.N. Mohapatra, 1976, Phys. Rev. D13, 104,

\bibitem{}
Branco, G.C., 1980, Phys. Rev. Lett. 44, 504.

\bibitem{}
Bryman, D.A. et al., 1972, Phys. Rev. lett. 28, 1469.

\bibitem{}
Bryman, D.A. and C. Picciotto, 1978, Rev. Mod. Phys. 50, 11.

\bibitem{}
Cahn, R.N. and H. Harari, 1980, Nucl. Phys. B176, 135.

\bibitem{}
Chakrabarti, J., M. Popovic and R.N. Mohapatra, 1980, Phys. Rev. b21, 3212.

\bibitem{}
Chaniowitz, M.S., J. Ellis and M.K. Gaillard, 1977, Nucl. Phys. B128, 506.

\bibitem{}
Chikashige, Y., G. Gelmini, R.D. Peccei and M. Roncadelli, 1980, Phys. Lett. 94B, 499.

\bibitem{}
Cheng, T.P. and L.F. Li, 1977a, Phys. Rev. Di6, 1425.

\bibitem{}
Cheng, T.P. and L.F, Li, 1977b, Phys. Rev. DI6, 1565.

\bibitem{}
Cheng, T.P. and L.F. Li, 1978, Phys. Rev. DI7, 2375.

\bibitem{}
Cheng, T.P. and L.F. Li, 1980a, Phys. Rev. Lett. 45, 1908.

\bibitem{}
Cheng, T.P. and L.F. Li, 1980b, Phys. Rev. D22, 2860.

\bibitem{}
Clark, A.R. et al., 1971, Phys. Rev. Lett. 26, 1667.

\bibitem{}
Cowsik, R. and J. McClelland, 1972, Phys. Rev. Lett. 29, 669.

\bibitem{}
Cowsik, R., 1980, Bartol Report BA-80-41.

\bibitem{}
Danby, G. et al., 1962, Phys. Rev. Lett. 9, 36.

\bibitem{}
Daum, M. et al., 1979, Phys. Rev. D20, 2692.

\bibitem{}
Davidson, A., M. Koca and K.C. Wali, 1979a, Phys. Lett. 86B, 47.

\bibitem{}
Davidson, A., M. Koca and K.C. Wall, 1979b, Phys. Rev. Lett. 43, 92.

\bibitem{}
Davidson, A. and K.C. Wali, 1980, Weizman Report WIxxxx80/42/Nov-ph.

\bibitem{}
Davis, R., Jr., 1955, Phys. Rev. 97, 766.

\bibitem{}
Davis, R., Jr., J. Evans and B. Cleveland, 1978, in Proc of "Neutrino 78" Conference, edited by E. Fowler.

\bibitem{}
DeRejula, A., H. Georgl and S.L. Glashow, 1975, Phys. Rev. D12, 147.

\bibitem{}
Derman, E., 1979, Phys. Rev. DI9, 317.

\bibitem{}
Deshpande, N.G., R.C. Hwa and P.D. Mannheim, 1979, Phys. Rev. D139, 2686.

\bibitem{}
Deshpande, N.G., 1981, Eugene Report 0ITS-160.

\bibitem{}
Diamant-Berger, A.M. et al., 1976, Phys. Lett. 628, 485.

\bibitem{}
Dimopoulos, S. and J. Ellis, 1980, CERN Report TH.2949-CERN.

\bibitem{}
Dolgov, A.D. and Ya.B. Zeldovich, 1981, Rev. Mod. Phys. 53, 1.

\bibitem{}
Ecker, G., W. Grimms and W. Konetschny, 1981, Nucl. Phys. 8177, 489.

\bibitem{}
Egger, J., 1980, Nucl. Phys. A335, 87.

\bibitem{}
Eichten, E. and K. Lane, 1980, Phys. Lett. 90B, 125.

\bibitem{}
Ellis, J., M.K. Gaillard, D.V. Nanopoulos and P. Sikivie, 1980, CERN Report TH.2938-CERN/LAPP TH.23.

\bibitem{}
Enz, C.D., 1957, Nuovo Cimento 6, 250.

\bibitem{}
Feinberg, G., 1958, Phys. Rev. 110, 1482.

\bibitem{}
Feinberg, G. and S. Weinberg, 1961a, Phys. Rev. Lett. 6, 381.

\bibitem{}
Feinberg, G. and S. Weinberg, 1961b, Phys. Rev. 123, 1439.

\bibitem{}
Frankel, S., 1974, in Muon Physics, edited by V.W. Huges and C.S. Wu (Academic, New York), vol, II.

\bibitem{}
Frere, J.-M., 1981, Nucl. Phys. B177, 389.

\bibitem{}
Furry, W.H., 1939, Phys. Rev. 56, 1184,

\bibitem{}
Gaillard, M.K. and B.W. Lee, 1974, Phys. Rev. bio, 897.

\bibitem{}
Gell-Mann, M., 1953, Phys. Rev. 92, 833.

\bibitem{}
Gell-Mann, M., P. Ramond and R. Slansky, 1979, in Supergravity, edited by D.Z. Freedman (North-Holland, Amsterdam).

\bibitem{}
Georgi, H. and D.V. Nanopoulos, 1979, Nucl. Phys. B155, 52.

\bibitem{}
Glashow, S., 1961, Nucl. Phys. 22, 579.

\bibitem{}
Glashow, S.L., J. tliopoulos and L. Maiani, 1970, Phys. Rev. D2, 1285.

\bibitem{}
Goldhaber, M. 1977, in Unification of Elementary Forces and Gauge Theories, edited by D. Cline and F. Mills (Academic, New York).

\bibitem{}
Greuling, E. and R.C. Whitten, 1960, Apn. Phys. 11, 510.- 51 ~

\bibitem{}
Gribov, V. and B. Pontecorvo, 1969, Phys. Lett. 28B, 495.

\bibitem{}
Gurr, H.S., W.R. Kropp, F. Reines and B. Meyer, 1967, Phys. Rev. 158, 1321,

\bibitem{}
Halprin, A., P. Minkowski, H. Primakoff and S.P. Rosen, 1976, Phys. Rev. D13, 2567.

\bibitem{}
Harari, H., 1979, SLAC Report SLAC-PUB-2363.

\bibitem{}
Haxton, W.C., G.J. Stephenson, Jr., and D. Strottman, 1979, in Abstracts of 8th Int. Conf. on High Energy Physics and Nuclear Structure, (TRIUMF, Vancouver) p. 137.

\bibitem{}
Herczeg, P., 1979, In Proc. of Kaon Factory Workshop, edited by M.K, Craddock (TRIUMF, Vancouver), p. 20.

\bibitem{}
Hill, E.L., 1957, Rev. Mod. Phys. 23 253.

\bibitem{}
Inami, T. and C.S. Lim, 1980, Univ. of Tokyo-Komba Report.

\bibitem{}
Jackiw, R., 1970, Lectures on Current Algebra and its Applications (Princeton Univ., Princeton).

\bibitem{}
Kalyniak, P. and J.N. Ng, 1981, TRIUMF Report TRI-PP-81-8.

\bibitem{}
Kamal, A.N. and J.N. Ng, 1979, Phys. Rev. D20, 2269.

\bibitem{}
Kane, G.L. and R. Thun, 1980, Michigan Report UM HE 80-8, to be published
in Phys. Lett.

\bibitem{}
Kang, K. and A.C. Rothman, 1980, Brown Univ. Report IPNO/TH 80-02.

\bibitem{}
Kobayashi, M. and T. Maskawa, 1973, Prog. Theor. Phys. bo, 652.

\bibitem{}
Konopinski, E.J. and H.M. Mahmoud, 1953, Phys, Rev. 32, 1045.

\bibitem{}
Konopinski, E.J., 1955, Los Alamos Report LAMS-1949,

\bibitem{}
Korenchenko, S.M. et al., 1975, Zh. Eksp. Teor. Fiz. 70, 3 

[Sov. Phys. JETP 43, 1 (1976)].

\bibitem{}
Kuo, T.K. and S.T. Love, 1980, Purdue Report.

\bibitem{}
Lahanas, A.B. and C.E. Vayonakis, 1979, Phys. Rev. DI19, 2158.

\bibitem{}
Langacker, P., 1980, SLAC Report SLAC-PUB-254l.


\bibitem{}
Lee, B.W. and R.E. Shrock, 1977a, Phys. Rev. D16, hh.

\bibitem{}
Lee, B.W. and S. Weinberg, 1977b, Phys. Rev. Lett. 39, 165.

\bibitem{}
Lee, T.D., 1973, Phys. Rev. D8, 1226.

\bibitem{}
Lee, T.D., 1974, Phys. Rep. SC, 143.

\bibitem{}
Lee, T.D. and C.N. Yang, 1960, Phys. Rev. Lett. 4, 307.

\bibitem{}
Lipkin, H.J., 1980, Argonne National Lab Report ANL-HEP-CP-80-65.

\bibitem{}
Lu, D.C. et al., 1980, Phys. Rev. Lett. 45, 1066.

\bibitem{}
Lubimov, V.A. et al., 1980, Phys. Lett. 94B, 266.

\bibitem{}
Luders, G., 1958, Nuovo Cimento 7, 171.

\bibitem{}
Ma, E. and A. Pramudita, 1980, Hawaii Report UH-511-420-80.

\bibitem{}
Mack, G., 1979, Munich Report MPI-PAE/PTh 41/79.

\bibitem{}
Maehara, T. and T. Yanagida, 1979, Prog. Theor. Phys. 61, 1434,

\bibitem{}
Maki, Z., M. Nakagawa and 5. Sakata, 1962, Prog. Theor. Phys. 28, 870.

\bibitem{}
Mann, A.K., 1981, Philadelphia Report.

\bibitem{}
Mannheim, P.D., 1978, Oregon Report OTIS-98.

\bibitem{}
Marciano, W.J. and A. Sanda, 1977a, Phys. Rev. Lett. 38, 1512.

\bibitem{}
Marciano, W.J. and A. Sanda, 1977b, Phys. Lett. 678, 303.

\bibitem{}
Marciano, W.J., 1978, in New Frontiers in High Energy Physics, edited by

A. Perlmutter and L.F. Scott (Plenum, New York).

\bibitem{}
Marciano, W.J., 1980, Rockefeller Report DoE/EY/2232B-207.

\bibitem{}
Marshak, R.E. and R.N. Mohapatra, 1980a, Phys. Lett. 91B, 222.

\bibitem{}
Marshak, R.E. and R.N. Mohapatra, 1980b, in Proc. of Orbis Scientiae, 1980, (Coral Gables, Florida).

\bibitem{}
Marshak, R.E., Riazuddin, and R.N. Mohapatra, 1980, in Proc. of Muon Physics/Facility Workshop, edited by J.A. Macdonald, J.N. Ng and A.S. Strathdee (TRIUMF, Vancouver).

\bibitem{}
McWilliams, B. and L.F. Li, 1980a, Carnegie-Mellon Report CO0-3066-146.

\bibitem{}
McWilliams, B. and 0. Shanker, 1980b, Phys. Rev. D22, 2853.

\bibitem{}
Mohapatra, R.N. and J.C. Patl, 1975, Phys. Rev. D11, 566.

\bibitem{}
Mohapatra, R.N. and R.E. Marshak, 1980a, Phys. Rev. Lett. 44, 1316.

\bibitem{}
Mohapatra, R.N. and G. Senjanovic, 1980b, Fermilab Report Pub-80/61-THY.

\bibitem{}
Nakagawa, M., H. Okonogi, S. Sakata and A. Toyoda, 1963, Prog. Theor.

\bibitem{}
Phys. 30, 727.

\bibitem{}
Ng, J.N., 1980, TRIUMF Report TRI-PP-80-34.

\bibitem{}
Ng, J.N., 1981, Phys. Lett. 998, 53.

\bibitem{}
Nishijima, K., 1955, Prog. Theor. Phys. 13, 285.

\bibitem{}
Nishijima, K., 1957, Phys. Rev. 108, 907.

\bibitem{}
Ong, C.L., 1980, Phys. Rev. D22, 2886.

\bibitem{}
Pati, J.C. and A. Salam, 1974, Phys. Rev. D10, 275.

\bibitem{}
Pati, J.C., 1978, Wisconsin Report 79-078.

\bibitem{}
Pati, J.C., 1979, Maryland Report 80-003.

\bibitem{}
Pauli, W., 1957, Nuovo Cimento 6, 204.

\bibitem{}
Perl, M.L. et al., 1975, Phys. Rev. Lett. 35, 1489.

\bibitem{}
Petkov, S.T., 1977, Yad. Fiz. 25, 641 (Sov. J. Nucl. Phys. 25, 340).

\bibitem{}
Pontecorvo, B., 1950, Helv. Phys. Acta. Suppl. 3, 97.

\bibitem{}
Pontecorve, B., 1957, Zh. Eksp. Teor. Fiz. 33, 549.

\bibitem{}
Pontecorvo, B., 1959, JETP 37, 1751.

\bibitem{}
Pontecorvo, B., 1967, Zh. Eksp. Teor. Fiz. 53, 17117 [soviet Physics

JETP 26, 984 (1968)1.

\bibitem{}
Pontecorvo, B., 1968, Phys. Lett. 268, 630.

\bibitem{}
Primakoff, H., 1952, Phys. Rev. 85, 888.

\bibitem{}
Primakoff, H. and S.P. Rosen, 1959, Rept. Prog. Phys. 22, 121.

\bibitem{}
Primakoff, H. and S.P. Rosen, 1969, Phys. Rev. 184, 1925.

\bibitem{}
Primakoff, H. and S.P. Rosen, 1972, Phys. Rev. D5, 1784,


\bibitem{}
Racah, G., 1937, Nuovo Cimento 14, 322.

\bibitem{}
Ramond, P., 1979, Caltech Report CALT-68-709.

\bibitem{}
Reines, F. and M.F, Crouch, 1974, Phys. Rev. Lett. 32, 493.

\bibitem{}
Reines, F., H.W. Sobel and E. Pasierb, 1980, Phys. Rev. Lett. 45, 1307.

\bibitem{}
Rosen, S.P. and H. Primakoff, 1965, In Alpha-, Beta-, and Gamma-Ray

\bibitem{}
Spectroscopy, edited by K. Siegbahn (North-Holland, Amsterdam).

\bibitem{}
Rosen, S.P., 1980, Purdue University Report.

\bibitem{}
Rosen, S.P. and B. Kayser, 1981, Phys. Rev, D23, 669.

\bibitem{}
Salam, A. and J. Ward, 1964, Phys. Lett. 13, 168.

\bibitem{}
Sanda, A.l., 1981, Rockefeller Report DoE/EY/2232B-217.

\bibitem{}
Sato, H., 1980, Phys. Rev. Lett. 45, 1997.

\bibitem{}
Schechter, J. and J.W.F. Valle, 1980, Phys. Rev. D22, 1227.

\bibitem{}
Schwartz, M., 1960, Phys. Rev. Lett. 4, 306.

\bibitem{}
Schwinger, J., 1957, Ann. Phys. 2, 407.

\bibitem{}
Segre, G. and H.A. Weldon, 1979, Univ. of Pennsylvania Report UPR-0125T.

\bibitem{}
Senjanovic, G. and R.N. Mohapatra, 1975, Phys. Rev. D12, 1502.

\bibitem{}
Shanker, O., 1979, Phys. Rev. D20, 1608.

\bibitem{}
Shanker, O., 1980, TRIUMF Report TRI-PP-80-32, to be published in Phys. Rev. D.

\bibitem{}
Shanker, O., 1981, TRIUMF Report TRI-PP-81-3, to be published in Nucl. Phys. 8,

\bibitem{}
Shrock, R.E. and S.B. Treiman, 1979, Phys. Rev. D19, 2148.

\bibitem{}
Shrock. R.E., 1980, Stony Brook Report ITP-SP-80-56.

\bibitem{}
Shvartsman, V., 1969, JETP Lett. 9, 184,

\bibitem{}
Simpson, J.J., 1981, Phys. Rev. D23, 649.

\bibitem{}
Steigman, G., D. Schramm and J. Gunn, 1977, Phys. Lett. 66B, 202.

\bibitem{}
Steigman, G., 1979, Ann. Rev. Nucl. Part. Sci. 23, 313.

\bibitem{}
Street, J.C. and E.C. Stevenson, 1937, Phys. Rev. 52, 1003.

\bibitem{}
Susskind, L., 1980. Phys. Rev. D20, 2619.

\bibitem{}
Taylor, J.C., 1976, Gauge Theories of Weak Interactions, 

(Cambridge Univ. Press, Cambridge).

\bibitem{}
t'Hooft, G., 1976, Phys. Rev. Lett. 37, 8.

\bibitem{}
Tonschek, B., 1949, Zeit. fur Physik, 125, 108.

\bibitem{}
Treiman, S.B., F. Wilczek and A. Zee, 1977, Phys. Rev. bis, 152.

\bibitem{}
Tremaine, S. and J. Gunn, 1979, Phys. Rev. Lett. 42, 407.

\bibitem{}
Tung, W.K., 1977, Phys. Lett. 67B, 52.

\bibitem{}
Vergados, J.D., 1981, Phys. Rev. D23, 703.

\bibitem{}
Weinberg, S. and G. Feinberg, 1959, Phys. Rev. Lett. 3, 111, erratum,

\bibitem{}
ibid 244,

\bibitem{}
Weinberg, S., 1967, Phys. Rev. Lett, 19, 1264.

\bibitem{}
Weinberg, S., 1972, Phys. Rev. D5, 1962.

\bibitem{}
Weinberg, S., 1976, Phys. Rev. D13, 974.

\bibitem{}
Weinberg, S., 1977a, Proc. of the 7th International Conference on High

\bibitem{}
Energy Phyeics and Nuclear Structure, (Zurich) p. 339.

\bibitem{}
Weinberg, S5., 1977b, Trans. of the New York Academy of Sciences, 38, 185.

\bibitem{}
Weinberg, S., 1978, Proce. of 19th International Conference on High Energy

\bibitem{}
Physics, Tokyo, edited by S. Homma, M. Kawaguchi and H. Miyazawa (Phys. Soc. of Japan, Tokyo, 1979).

\bibitem{}
Weinberg, S., 1979, Phys. Rev. Lett. 43, 1566.

\bibitem{}
Wilczek, F. and A. Zee, 1979a, Phys. Rev, Lett, 42, 421,

\bibitem{}
Wilczek, F. and A. Zee, 1979b, Phys. Rev. Lett. 43, 1566.

\bibitem{}
Willis, S.E. et al., 1980, Phys. Rev. Lett. 4h, 522.

\bibitem{}
Witten, E., 1980, Phys. Lett. 31B, 81.

\bibitem{}
Wolfenstein, L., 1964, Phys. Rev. Lett. 13, 562.

\bibitem{}
Wolfenstein, L., 1978, Phys. Rev. D18, 958.

\bibitem{}
Wolfenstein, L., 1980a, Nucl. Phys. B175, 98.

\bibitem{}
Wolfenstein, L., 1980b, Lawrence Berkeley Lab Report C00-3066-160.

\bibitem{}
Zee, A., 1980, Phys. Lett. 93B, 389.


\end{thebibliography} 

\end{document}


