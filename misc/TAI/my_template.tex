\documentclass[journal]{IEEEtai}

\usepackage[colorlinks,urlcolor=blue,linkcolor=blue,citecolor=blue]{hyperref}

\usepackage{color,array}

\usepackage{graphicx}

%% \jvol{XX}
%% \jnum{XX}
%% \paper{1234567}
%% \pubyear{2020}
%% \publisheddate{xxxx 00, 0000}
%% \currentdate{xxxx 00, 0000}
%% \doiinfo{TQE.2020.Doi Number}

\newtheorem{theorem}{Theorem}
\newtheorem{lemma}{Lemma}
\setcounter{page}{1}
%% \setcounter{secnumdepth}{0}


\begin{document}


\title{Generative AI and the Riemann zeta zero distribution} 


\author{Placeholder, for double blind review
\thanks{This paragraph of the first footnote will contain the date on which you submitted your paper for review. It will also contain support information, including sponsor and financial support acknowledgment. For example, ``This work was supported in part by the U.S. Department of Commerce under Grant BS123456.'' }
\thanks{Placeholder, for double blind review}
\thanks{This paragraph will include the Associate Editor who handled your paper.}}

\markboth{Journal of IEEE Transactions on Artificial Intelligence, Vol. 00, No. 0, Month 2020}
{First A. Author \MakeLowercase{\textit{et al.}}: Bare Demo of IEEEtai.cls for IEEE Journals of IEEE Transactions on Artificial Intelligence}

\maketitle

\begin{abstract}
The Transformer model of Generative AI introduced attention mechanisms that has had important successes in natural language processing.  We applied it for predicting the distribution of Riemann zeta zero counts on consecutive Gram intervals. The results are very good. We get accuracies of $0.998$ in predicting a sequence of ten consecutive zero counts. The input for the prediction is a sequence of the preceeding ten zero counts on sequential Gram intervals. We tested with two ranges of Riemann zeta zeros, t (height along critical axis for Riemann zeta function) = $10^{12}$ and t = $10^{28}$. The $0.998$ comes about because the intervals with $3$ or more zeros are rare. With special training for rare events, we can get essentially completely accurate prediction. One surprising result is that the model trained on a small amount of data at $10^{28}$ accurately predicts the behaviour $16$ orders of magnitude away, at $10^{12}$. The high prediction accuracy was unexpected. We present arguments that the 100\% accurate predictions cannot continue indefinitely. Our study shows that applying the technique to more complex problems has great promise. We have used very minimal computer resources compared to typical models in language applications. With access to better resources, we can attack much more important problems. 
\end{abstract}

\begin{IEEEImpStatement}
Our application of the Transformer model, renowned for its attention mechanisms in natural language processing, marks a significant breakthrough in predicting the distribution of Riemann zeta zero counts on consecutive Gram intervals. Achieving accuracies as high as $0.998$ in forecasting sequences of ten consecutive zero counts underscores the model's robustness and effectiveness in this specialized domain. By focusing on rare events — intervals with three or more zeros — and employing tailored training strategies, we achieved near-perfect predictions, showcasing the Transformer's adaptability beyond traditional linguistic tasks. Moreover, our study demonstrates promising potential for applying these techniques to more intricate problems, leveraging minimal computational resources compared to standard language models. With enhanced computational capabilities, we anticipate tackling even more impactful challenges, highlighting the transformative impact of advanced AI methodologies in mathematical research and beyond.

\end{IEEEImpStatement}

\begin{IEEEkeywords}
Attention, Generative AI, Riemann zeta, Transformer Architecture
\end{IEEEkeywords}



\section{Introduction}

\IEEEPARstart{T}{he}  Transformer model~\cite{vaswani2017attention} introduced attention mechanisms that revolutionized natural language processing (NLP).
Recent advancements \cite{radford2019language,brown2020language} include significant improvements in few-shot learning capabilities.
The effectiveness of transfer learning using a unified text-to-text Transformer approach is studied in \cite{raffel2019exploring}.
Transformer-XL, described in \cite{dai2019transformerxl}, extends the Transformer architecture to handle longer contexts in language modeling tasks.
Transformer-XH, proposed by \cite{zhang2019transformerxh}, addresses multi-hop question answering with explanation capabilities. 

Given the success of generative AI in extracting patterns in NLP, there is potential to apply similar techniques to challenging problems such as the Riemann zeta function.
Machine learning has been used in mathematical applications. An early example is  Ref.~\cite{osneural}. This was followed by \cite{he2022sato}.

\section{Background}

To make this more concrete, in Table~\ref{tab:sequence} we show a sequence of zero counts on consecutive Gram intervals. The sequence of counts has been broken up into separate rows, merely for convenience in presentation. The problem we solve is: given a sequence of ten such intervals, predict the zero counts on the next set of ten successive intervals.

what problem? why? simplified view.
For those who do not wish to get too deep into Riemann zeta theory, just ignore all the discussion about the theory, and understand that we have a sequence of counts, and we wish to see how well Generative AI does in predicting the pattern underlying the sequence of counts. 

\begin{table}
\centering \(\begin{array}{ccccccccccccccc}
\hline
2& 0& 1& 2& 0& 1& 1& 1& 0& 2& 1& 0& 1& 2& 1 \\
2& 0& 1& 1& 1& 2& 0& 1& 1& 1& 0& 2& 1& 1& 1 \\
0& 2& 1& 2& 1& 0& 1& 1& 1& 1& 1& 1& 1& 1& 1 \\
1& 1& 1& 1& 1& 1& 1& 0& 1& 2& 1& 1& 1& 1& 0 \\
2& 1& 1& 0& 3& 1& 0& 2& 1& 1& 0& 1& 1& 1& 1 \\
\hline
\end{array}\)
\caption{Count of zeros on consecutive Gram intervals (shown on multiple lines for convenience).} 
\label{tab:sequence}
\end{table}

\subsection{Theta Equation}
xxx

Define 


\section{Methodology}

history. start with $10^{12}$. First zero   at $10^{12} + 244.158907$. reached 0.998, plateaued. realized problem with 3s. went to $10^{28}$. only 3s enough. covers large orders of magnitude.


\section{Analysis}

Use 
Table of training iter and validation, ranges

\begin{figure}
\centerline{\includegraphics[width=18.5pc]{Figure1.png}}
\caption{Magnetization as a function of applied field. Note that ``Fig.'' is abbreviated. There is a period after the figure number, followed by two spaces. It is good practice to explain the significance of the figure in the caption.}
\end{figure}




\section{Conclusion and Future Work}

A conclusion section is not required. Although a conclusion may review the main points of the paper, do not replicate the abstract as the conclusion. A conclusion might elaborate on the importance of the work or suggest applications and extensions.






\section*{References}

\begin{thebibliography}{34}

\bibitem{vaswani2017attention}
A. Vaswani, N. Shazeer, N. Parmar, J. Uszkoreit, L. Jones, A. N. Gomez, Ł. Kaiser, and I. Polosukhin, ``Attention is all you need",  \emph{Advances in Neural Information Processing Systems (NeurIPS)}, vol. 30, pp. 5998--6008, 2017.

\bibitem{radford2019language}
A. Radford, J. Wu, R. Child, D. Luan, D. Amodei, and I. Sutskever, ``Language models are unsupervised multitask learners", \emph{OpenAI Blog}, vol. 1, no. 8, p. 9, 2019.

\bibitem{brown2020language}
T. B. Brown, B. Mann, N. Ryder, M. Subbiah, J. Kaplan, P. Dhariwal, A. Neelakantan, P. Shyam, G. Sastry, A. Askell, et al., ``Language models are few-shot learners",  \emph{Advances in Neural Information Processing Systems (NeurIPS)}, vol. 33, 2020.

\bibitem{raffel2019exploring}
C. Raffel, N. Shazeer, A. Roberts, K. Lee, S. Narang, M. Matena, Y. Zhou, W. Li, and P. J. Liu, ``Exploring the limits of transfer learning with a unified text-to-text transformer", \emph{arXiv preprint arXiv:1910.10683}, 2019.

\bibitem{dai2019transformerxl}
Z. Dai, Z. Yang, Y. Yang, J. Carbonell, Q. V. Le, and R. Salakhutdinov, ``Transformer-XL: Attentive language models beyond a fixed-length context",  \emph{Proceedings of the 57th Annual Meeting of the Association for Computational Linguistics (ACL)}, pp. 2978--2988, 2019.

\bibitem{zhang2019transformerxh}
Y. Zhang, Z. Gan, R. Henao, D. Shen, and L. Carin, ``Transformer-XH: Multi-hop question answering with explanation",  \emph{Advances in Neural Information Processing Systems (NeurIPS)}, vol. 32, 2019.

\bibitem{osneural} O. Shanker, ``Neural Network prediction of Riemann zeta zeros'',
\emph{Advanced Modeling and Optimization}, vol. 14, 717-728, 2012. \url{https://www.researchgate.net/publication/273945970_Neural_Network_prediction_of_Riemann_zeta_zeros}.

\bibitem{he2022sato} Y. H. He, K. H. Lee, and T. Oliver, ``Machine-learning the Sato–Tate conjecture'', \emph{Journal of Symbolic Computation}, vol. 111, 61-72, 2022.

\end{thebibliography}






\begin{IEEEbiography}{First A. Author}{\space}   placeholder, for double blind review
\end{IEEEbiography}


\end{document}
